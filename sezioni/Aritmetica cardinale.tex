\section{Aritmetica cardinale}

\begin{definition}[Cardinali]
	Diciamo che $\kappa$ è un \vocab{cardinale} se $\kappa$ è un ordinale iniziale.
\end{definition}

\begin{notation}[Cardinali successori]
	Indichiamo con la notazione:
	\[ \kappa = \lambda^+ \Mydef \text{$\kappa$ è il minimo cardinale $>\lambda$}
		\]
	ad esempio, $\omega_\alpha^+ = \omega_{\alpha + 1}$, ovvero l'ordinale iniziale successivo [che quindi ha cardinalità strettamente più grande].
\end{notation}

Possiamo definire le operazioni di prodotto, somma e potenza cardinale (per il \hyperref[buon_ordinamento]{teorema del buon ordinamento} le operazioni sulle cardinalità e sugli ordinali iniziali corrispondono), per cui se $\kappa = \omega_\alpha$, $\lambda = \omega_\beta$ e $\mu = \omega_\gamma$ scriviamo:
\[ \kappa =  \lambda + \mu \qquad \kappa = \lambda \cdot \mu \qquad \kappa = \lambda^\mu
	\]
per rispettivamente:
\[ \aleph_\alpha = \aleph_\beta + \aleph_\gamma \qquad \aleph_\alpha = \aleph_\beta \cdot \aleph_\gamma \qquad \aleph_\alpha = \aleph_\beta^{\aleph_\gamma}
	\]
Occorre fare attenzione a non confondere le operazioni \textcolor{red}{cardinali} con quelle \textcolor{MidnightBlue}{ordinali}. In questa sezione ci occupiamo di operazioni \textcolor{red}{cardinali}.\\
Siccome ad ogni cardinalità corrisponde un unico ordinale iniziale [per \hyperref[ax9]{AC}], che abbia quella cardinalità, possiamo scrivere:
\[ \kappa = |X| \Mydef \text{$\kappa$ è un cardinale e $|\kappa| = |X|$}
	\]
Cioè stiamo commettendo un piccolo abuso di notazione (o di definizione) e dicendo che la definizione di cardinalità può essere in realtà quella dell'ordinale iniziale a lui associato, in realtà questa cosa non è sbagliata, nel senso che alla fine, tutte le cardinalità degli insiemi (per AC) sono ordinali iniziali, cioè vi è sempre una bigezione tra ordinale iniziale e insieme (se hanno la stessa cardinalità, ma la cardinalità, come detto in precedenza non è qualcosa di per sé, ma è una definizione che 
considera sempre due insiemi), quello che stiamo facendo con i cardinali non è altro che fissare una notazione per dei ``rappresentanti privilegiati'' delle cardinalità, che sono appunto gli ordinali iniziali.\\
Per questi ultimi valgono tutte le proprietà viste sulle cardinalità, che quindi ci danno le operazioni cardinali (che NON sono un'estensione di quelle ordinali, ma altre operazioni definite diversamente e soltanto tra ordinali iniziali, perché entrano in gioco le proprietà viste sulle cardinalità).\footnote{Osservare che nel caso degli ordinali [iniziali] finiti le operazioni ordinali coincidono con quelle cardinali ed entrambe coincidono con le operazioni definite per ricorsione numerabile su $\omega$, questo 
perché gli elementi di $\omega$ sono i loro stessi rappresentanti canonici di ordinali e sono tutti ordinali iniziali (per tutte le cardinalità finite).}
\subsection{Somme e prodotti infiniti}

\begin{definition}[Somme e prodotti infiniti]
	Sia $I$ un insieme e $\{\kappa_i\}_{i \in I}$ una famiglia di cardinali. Definiamo la \vocab{somma} e il \vocab{prodotto sulla famiglia} indicizzata da $I$ come:\footnote{La somma non è altro che una naturale estensione della somma tra cardinalità, per la somma di due cardinalità, in altre parole, stiamo rendendo disgiunti tutti gli insiemi e poi li stiamo unendo. Osservare anche che la definizione di prodotto data non usa AC.}
	 \begin{align*}
		\sum_{i \in I} \kappa_i &\Mydef\left\lvert \bigcup\{\kappa_i \times \{i\} | i \in I\}\right\rvert \\
		\prod_{i \in I} \kappa_i &\Mydef \left\lvert \left\{f : I \rightarrow \sup_{i \in I} \kappa_i \m  \forall i \in I \; f(i) \in \kappa_i\right\}\right\rvert 
	 \end{align*}
	(osservare che il sup di una famiglia di cardinali, essendo ordinali iniziali, è dato dall'unione di questi ultimi\footnote{Inoltre, notare anche come sapessimo già fare prodotti infiniti degli stessi insiemi, considerando le funzioni dall'uno all'altro, ma ciò valeva solo volendo fare un prodotto infinito di uno stesso insieme, per un numero di volte dato dall'insieme di arrivo
	(e.g. ${}^\omega A = \underbrace{A \times \ldots \times A}_{\text{$\omega$ volte}.}$).}).
\end{definition}

\textcolor{MidnightBlue}{Formalmente la famiglia $\{\kappa_i\}_{i \in I}$ è una funzione $f$ con $\Dom(f) = I$ e $\forall i \in I \; f(i)$ è un cardinale, $\kappa_i$ è un'abbreviazione per $f_i$.}

\begin{remark}[Prodotti cartesiani infiniti]
	La definizione sopra generalizza quella di prodotto cartesiano data, a prodotto cartesiano di una famiglia qualunque di insiemi qualunque. Nel caso finito c'è una bigezione tra le due definizioni, infatti, data una $f$ che va da $I$ all'unione $\bigcup \{X_i\}_{i \in I}$ [con $I$ finito] di una 
	famiglia di insiemi, tale $f$ per definizione fissa un elemento in ogni insieme, $f(i) \in X_i$, $\forall i \in I$, a questo punto, possiamo scrivere le immagini di $f(i)$ in una $|I|$-upla ordinata [secondo l'ordinamento su $I$], ed ottenere l'elemento del prodotto cartesiano finito voluto.\\
	Viceversa una $|I|$-upla definisce completamente una mappa da $I$ a $\bigcup\{X_i\}_{i \in I}$, infatti basta prendere come $f(i)$ l'$i$-esima componente della tupla, e tale componente sta in $X_i$, dunque si ottiene proprio una funzione che rispetta la proprietà richiesta.
\end{remark}

\begin{note}[Somma di cardinali disgiunti]
	Sia $\{X_i\}_{i \in I}$ una famiglia di insiemi a due a due \textbf{disgiunti}, e sia $\forall i \in I \; \kappa_i = |X_i|$\footnote{Se non sono già dati i cardinali c'è bisogno di AC.} allora:
	\[ \sum_{i \in I} \kappa_i = \left\lvert \bigcup\{X_i | i \in I\}\right\rvert 
		\]
	ovvero se gli insiemi a cui sono associati i cardinali sono disgiunti, la somma è semplicemente la cardinalità dell'unione senza bisogno di costruire l'unione disgiunta.\footnote{Questa proprietà così naturale nel caso finito, richiede AC per essere valida nel caso generale e dare una coerenza alle nostre definizioni, da qui l'importanza di assumere scelta per estendere le operazioni cardinali al caso infinito.}
\end{note}

\begin{proof}
	(Richiede \hyperref[ax9]{AC})\\
	Per ogni $i \in I$ scegliamo una bigezione $f_i : \kappa_i \times \{i\} \rightarrow X_i$, con $|\kappa_i \times \{i\}| = |\kappa_i| = |X_i|$\footnote{Per la precisione consideriamo $B = \{B_i\}_{i \in I}$, con $B_i = \{f_i : \kappa_i \times \{i\} \to X_i | \;\text{$f_i$ bigezione}$\},
	poiché per ogni $i \in I$ vale l'uguaglianza tra cardinalità scritta sopra, allora nessuno dei $B_i$ è vuoto, ovvero $\emptyset \not \in B$, dunque possiamo usare AC e ottenere una funzione $\phi : B \to \bigcup B : B_i \mapsto \phi(B_i) \in B_i$, che fissa per ogni insieme $B_i$ una sua bigezione.
	Osservare che usiamo AC per l'arbitrarietà di $|I|$, se fosse finito non avremmo bisogno di AC per fissare le bigezioni (per questo questa proprietà non ha bisogno di AC nel caso finito), ma nel caso infinito non possiamo fare altrimenti.}, allora si ha che:
	\[ f : \bigcup \{\kappa_i \times\{i\} | i \in I\} \rightarrow \bigcup \{X_i | i \in I\} : (a,i) \mapsto f_i(a,i)\,\footnote{Typo Mamino.}
		\]
	ossia $f = \bigcup\{f_i | i \in I\}$, ed è ben definita perché gli insiemi sono disgiunti inoltre è una bigezione perché unione di bigezioni definite su insiemi disgiunti in arrivo.
	Abbiamo quindi una bigezione tra l'insieme usato per la definizione di somma di cardinalità e l'unione degli insiemi corrispondenti a tali cardinalità, dunque la cardinalità dell'unione degli insiemi è proprio uguale alla somma delle cardinalità degli insiemi.
\end{proof}

\begin{note}[Disuguaglianza di inclusione-esclusione nel caso generale]
	Sia $\{X_i\}_{i \in I}$ una famiglia di insiemi non necessariamente a due a due disgiunti, e sia $\forall i \in I \; \kappa_i = |X_i|$\footnote{Come prima, se non sono già dati i cardinali c'è bisogno di AC.} allora in generale vale la disuguaglianza di inclusione-esclusione,
	ovvero la cardinalità dell'unione della famiglia è minore o uguale alla somma delle cardinalità degli elementi della famiglia:
	\[	\left\lvert\bigcup\{X_i | i \in I\} \right\rvert \leq \sum_{i \in I} \kappa_i
		\]
	Inoltre la definizione di prodotto data è ben posta indipendentemente dagli insiemi di cui si considera la cardinalità:
	\[ \prod_{i \in I} \kappa_i = \left\lvert\left\{f : I \to \bigcup_{i \in I} X_i \m \forall i \in I \; f(i) \in X_i\right\}\right\rvert 
		\]
\end{note}

\begin{proof}
	(Richiede \hyperref[ax9]{AC})\\
	Per la prima disuguaglianza si può fare come nel caso precedente, ovvero $\forall i \in I$ abbiamo $|\kappa_i \times \{i\}| = |X_i|$, per cui (usando AC nel caso generale) possiamo fissare una bigezione $f_i$, e considerare $f = \bigcup_{i \in I} f_i$, data da:
	\[ f : \bigcup\{\kappa_i \times \{i\} | i \in I\} \to \bigcup \{X_i | i \in I\} : (a,i) \mapsto f_i(a)
		\]
	in questo caso abbiamo che $f$ è unione di bigezioni, ma in arrivo gli insiemi non sono tutti necessariamente disgiunti, quindi ci può essere un elemento che ha più controimmagini, pertanto $f$ è soltanto surgettiva e tale surgettività ci dà:
	\[ \sum_{i \in I} \kappa_i = \left\lvert \bigcup\{\kappa_i \times \{i\} | i \in I\}\right\rvert \geq \left\lvert \bigcup \{X_i | i \in I\}\right\rvert 
		\]
	Per la seconda osservazione possiamo ancora una volta fissare per ogni $i \in I$ le bigezioni $g_i : \kappa_i \rightarrow X_i$ (con AC) e mandare ogni funzione del primo insieme in una funzione che valutata in ogni $i$ restituisce il valore di $f(i)$ trasportato in $X_i$ mediante la bigezione fissata per quell'$i$:
	\begin{align*}
		g : \left\{f : I \rightarrow \sup_{i \in I} \kappa_i \m \forall i \in I \; f(i) \in \kappa_i\right\} &\rightarrow \left\{f : I \rightarrow \bigcup \{X_i | i \in I\} \m \forall i \in I \; f(i) \in X_i\right\} \\
																								     f &\mapsto g(f)(i) = g_i(f(i))
	\end{align*}
	tale mappa è bigettiva, perché se due mappe in arrivo coincidono su tutti gli $i \in I$, essendo le $g_i$ bigezioni, si ha che le mappe in partenza sono uguali, inoltre presa una mappa $h$ nell'insieme in arrivo, si può ottenere la sua controimmagine costruendo una
	funzione fatta da $g_i^{-1}(h(i))$, per ogni $i \in I$.
\end{proof}

\begin{remark}[Prodotto infinito di potenze]
	Vale la proprietà del prodotto di potenze\footnote{$a^{n+m} = a^na^m$.} anche nel caso di prodotti arbitrari:
	\[ \kappa^{\sum_{i \in I}\lambda_i} = \prod_{i \in I} \kappa^{\lambda_i}
		\]
\end{remark}

\begin{proof}
	Un elemento dell'insieme al LHS è una funzione $f : \sum_{i \in I} \lambda_i = \bigcup\{\lambda_i \times \{i\} | i \in I\} \rightarrow \kappa$,
	tale funzione può essere identificata in una funzione:
	\[ \prod_{i \in I}\kappa^{\lambda_i}\ni\widetilde{f} : I \rightarrow \bigcup\{\kappa^{\lambda_i} | i \in I\} : i \mapsto \widetilde{f}_i \in \kappa^{\lambda_i}
		\]
	ovvero mandiamo $f$ in una funzione che associa ad ogni $i \in I$ la funzione $\widetilde{f}_i := f(\cdot,i) \in \kappa^{\lambda_i}$. È abbastanza immediato verificare che è iniettiva, infatti se due funzioni in arrivo coincidono praticamente le funzioni in partenza coincidono su tutte le coppie di elementi 
	in $\bigcup\{\lambda_i \times \{i\} | i \in I\}$, si verifica inoltre che è anche surgettiva.
\end{proof}

\begin{exercise}[Proprietà delle operazioni cardinali]
	Valgono le proprietà ragionevoli per le operazioni tra cardinali, ad esempio:
	\begin{itemize}
		\item se $\forall i \in I \; \kappa_i \leq \lambda_i$: \textcolor{orange}{(compatibilità con l'ordinamento - versione infinita)}
		\[ \sum_{i \in I} \kappa_i \leq \sum_{i \in I} \lambda_i \qquad \prod_{i \in I} \kappa_i \leq \prod_{i \in I} \lambda_i
			\]
		\item $\forall i \in I \; \kappa_i \leq \sum_{i \in I} \kappa_i$ \textcolor{orange}{(corollario di quella sopra)}
		\item Se $I = I_1 \sqcup I_2$: \textcolor{orange}{(associatività infinita)}
		\[ \sum_{i \in I} \kappa_i = \left(\sum_{i \in I_1} \kappa_i\right) + \left(\sum_{i \in I_2} \kappa_i\right) \qquad \prod_{i \in I}\kappa_i = \left(\prod_{i \in I_1} \kappa_i\right)\cdot\left(\prod_{i \in I_2} \kappa_i\right)
			\]
		\item \textcolor{orange}{(compatibilità tra le definizioni di operazioni cardinali)}:
		\[ \sum_{i \in \kappa} \lambda = \kappa \cdot \lambda \qquad \prod_{i \in \kappa} \lambda = \lambda^\kappa
			\]
		\item \textcolor{orange}{(prodotto di potenze)}:
		\[ \left(\prod_{i \in I}\kappa_i\right)^\lambda = \prod_{i \in I} \kappa_i^\lambda
			\]
	\end{itemize}
\end{exercise}


\pagebreak
\begin{proposition}[Somma infinita di cardinali]
	Supponiamo che $\{\kappa_i\}$, per $i \in I$ sia una famiglia di cardinali diversi da 0, allora:
	\[ \sum_{i \in I}\kappa_i = |I| \cdot \sup_{i \in I} \kappa_i = \max\left(|I|, \sup_{i \in I}\kappa_i\right)
		\]
\end{proposition}

\begin{proof}
	La seconda uguaglianza deriva dal fatto che le cardinalità sono tutte aleph, per cui vale la proprietà vista per il prodotto tra aleph. Per la prima dimostriamo le due disuguaglianze come segue [e concludiamo con Cantor-Bernstein].
	\begin{itemize}
		\item[$\boxed{\leq}$] Deriva facilmente dalle proprietà viste sopra:
		\[ \sum_{i \in I} \kappa_i \leq \sum_{i \in I} \sup_{i \in I}\kappa_i = |I| \cdot \sup_{i \in I} \kappa_i
			\]
		dove appunto, la prima disuguaglianza è esattamente la compatibilità della somma con l'ordinamento dei cardinali, mentre la seconda uguaglianza è la compatibilità tra le definizioni delle operazioni.
		\item[$\boxed{\geq}$] Siccome $|I| \cdot \sup_{i \in I} \kappa_i$ è il massimo fra $|I|$ e $\sup_{i \in I} \kappa_i$ [lo abbiamo già visto all'inizio], basta verificare che la somma al LHS è maggiore di entrambi separatamente.
		$\sum_{i \in I}\kappa_i \geq \sum_{i \in I} 1 = |I|$, dove la prima disuguaglianza è la compatibilità prodotto-ordinamento [applicata a cardinali $>0$ per ipotesi] e la seconda quella delle operazioni. L'altra disuguaglianza segue osservando che:
		\[ \forall j \in I \; \sum_{i \in I} \kappa_i \geq \kappa_j
			\]
		cioè tutta la famiglia è più piccola della somma, quindi deve esserlo anche il sup di tale famiglia.
	\end{itemize}
\end{proof}

\subsection{Teorema di König}

\begin{proposition}[Teorema di \href{https://en.wikipedia.org/wiki/Gyula_K\%C5\%91nig}{\textcolor{purple}{König}}]
	Se $\forall i \in I \; \kappa_i < \lambda_i$ allora:
	\[ \sum_{i \in I} \kappa_i < \prod_{i \in I} \lambda_i
		\]
\end{proposition}

\begin{proof}
	Dimostriamo che non può valere il $\geq$. Siano:
	\[ A := \bigcup\{\kappa_i \times \{i\} | i \in I\} \qquad B := \left\{f : I \rightarrow \sup_{i \in I} \lambda_i \m \forall i \in I \; f(i) \in \lambda_i\right\} 
		\]
	gli insiemi le cui cardinalità definiscono rispettivamente il prodotto della famiglia $\{\lambda_i\}_{i \in I}$ e la somma della famiglia $\{\kappa_i\}_{i \in I}$.
	Data una qualunque funzione $f : A \rightarrow B$\footnote{Typo di Mamino.} dobbiamo dimostrare che non può essere surgettiva (ovvero $\neg |B| \geq |A|$, cioè non è vero che la somma è maggiore o uguale al prodotto).
	Consideriamo la famiglia di funzioni tra i cardinali $\kappa_i$ e $\lambda_i$ definita via $f$ da:
	\[ f_i : \kappa_i \to \lambda_i : \alpha \mapsto (\underbrace{f(\alpha,i)}_{\in B})(i) \in \lambda_i
		\]
 	siccome per ipotesi $\kappa_i < \lambda_i$ le funzioni $f_i$ non possono essere surgettive per ogni $i \in I$, e ciò si traduce nel fatto che, per ogni $i \in I$, $\lambda_i \setminus\Imm(f_i) \ne \emptyset$.\\
	Per quanto detto $\emptyset \not \in \{\lambda_i \setminus\Imm(f_i)\}_{i \in I}$, dunque possiamo usare AC per fissare fissare un elemento in ciascun insieme $\lambda_i \setminus\Imm(f_i)$, in particolare possiamo
	scrivere una funzione che associa l'indice $i$ in $I$ al corrispettivo elemento fissato in $\lambda_i \setminus \Imm(f_i)$ come segue:
	\[ g : I \rightarrow \sup_{i \in I} \lambda_i : i \mapsto g(i) \in \lambda_i \setminus \Imm(f_i)
		\]
	osserviamo che $g \in B$ in quanto $g(i) \in \lambda_i$ per ogni $i \in I$, inoltre vale che che $g \not \in \Imm(f)$. Se, per assurdo, $g \in \Imm(f)$, allora [ricordando che $f$ è definita sulle coppie di $A$]
	abbiamo $g = f(\alpha,i)$, per qualche $(\alpha, i) \in A$, da cui [per estensionalità per funzioni]: 
	\[ g(i) = f(\alpha,i)(i) \overset{\text{def. $f_i$}}{=} f_i(\alpha) \in \Imm(f_i) \; \lightning
		\]
	che è assurdo in quanto $g(i) \in \lambda_i \setminus \Imm(f_i)$ per definizione. Abbiamo quindi verificato che c'è sempre un elemento di $B$, $g \not \in \Imm(f)$, che rende falsa la surgettività di una qualunque funzione.
	Poiché per AC le cardinalità sono totalmente ordinate, deve quindi valere necessariamente che all'inizio si ha $|A| < |B|$.
\end{proof}

\begin{example}[Disuguaglianza di Cantor]
	Osserviamo che applicando il teorema di König sui cardinali 1 e 2, sommati su una famiglia $\kappa$, si ottiene:
	\[ \kappa = \sum_{i \in \kappa} 1 \;\textcolor{red}{<}\; \prod_{i \in \kappa} 2 = 2^\kappa
		\]
	(dove le uguaglianze laterali sono la compatibilità delle definizioni delle operazioni), ovvero proprio il \hyperref[cantor]{teorema di Cantor} dimostrato in precedenza, che ora diventa un caso particolare del teorema di König.
\end{example}

\begin{example}[$2^{\aleph_0} \ne \aleph_\omega$]
	Osserviamo che vale:
	\[ \aleph_\omega = \max\left(\aleph_0, \sup_{i \in \omega} \aleph_i\right) = \sum_{i \in \omega} \aleph_i \;\textcolor{red}{<}\; \prod_{i \in \omega} \aleph_{i + 1} \leq \prod_{i \in \omega} \aleph_\omega = \aleph_\omega^{\aleph_0}
		\]
	dove abbiamo usato che $\aleph_i < \aleph_{i + 1}$ per la definizione ricorsiva della funzione degli aleph. Questa cosa ci permette di osservare che, se valesse $2^{\aleph_0} = \aleph_\omega$, allora:
	\[ 2^{\aleph_0} = \aleph_\omega < \aleph_\omega^{\aleph_0} = (2^{\aleph_0})^{\aleph_0} = 2^{\aleph_0} \; \lightning
		\]
\end{example}

\begin{exercise}[Facile]
	Se $2^{\aleph_0} = \aleph_{41}$, allora $\aleph_{41}^{\aleph_0} = \aleph_{41}$.
\end{exercise}

\begin{soln}
	Basta sfruttare che il prodotto di numerabili è numerabile, infatti:
	\[ \aleph_{41}^{\aleph_0} \overset{\text{Hp.}}{=} (2^{\aleph_0})^{\aleph_0} = 2^{\aleph_0} \overset{\text{Hp.}}{=} \aleph_{41}
		\]
\end{soln}

\begin{exercise}[Difficile]
	Se $2^{\aleph_0} = \aleph_{41}$, allora $\aleph_{42}^{\aleph_0} = \aleph_{42}$.
\end{exercise}

\begin{soln}
	La soluzione più breve usa la formula di Hausdorff, che vederemo alla fine del capitolo (sarebbe $(\kappa^+)^{\lambda} = \kappa^{\lambda} \cdot \kappa^+$), per la quale:
	\[ \aleph_{42}^{\aleph_0} \overset{\text{Hausdorff}}{=} \aleph_{41}^{\aleph_0} \cdot \aleph_{42} \overset{\text{Hp.}}{=} (2^{\aleph_0})^{\aleph_0} \cdot \aleph_{42} = 2^{\aleph_0} \cdot \aleph_{42} \overset{\text{Hp.}}{=} \aleph_{41} \cdot \aleph_{42} = \aleph_{42}
		\]
\end{soln}

\begin{exercise}[Pure peggio]
	Se $2^{\aleph_0} = \aleph_{1}$\footnote{Che è l'\vocab{ipotesi del continuo} (\vocab{CH}).}, allora $\aleph_{n+1}^{\aleph_0} = \aleph_{n+1}$, per $n \in \omega$.
\end{exercise}

\begin{soln}
	Si procede per induzione numerabile e usando la formula di Hausdorff.
	\begin{itemize}
		\item[$\boxed{\text{caso 0}}$] In questo caso $\aleph_1^{\aleph_0} = (2^{\aleph_0})^{\aleph_0} = 2^{\aleph_0} = \aleph_1$.
		\item[$\boxed{\text{caso successore}}$] Assumiamo $\aleph_{n+1}^{\aleph_0} = \aleph_{n+1}$ e dimostriamo che $\aleph_{n+2}^{\aleph_0} = \aleph_{n+2}$. Utilizzando la formula di Hausdorff:
		\[ \aleph_{n+2}^{\aleph_0} = \aleph_{n+1}^{\aleph_0} \cdot \aleph_{n+2} \overset{\text{Hp. indutt.}}{=} \aleph_{n+1} \cdot \aleph_{n+2} = \aleph_{n+2}
			\]
	\end{itemize}
\end{soln}

\pagebreak
\subsection{Cofinalità}

\begin{definition}[Cofinalità - v.1]
	Dato un cardinale infinito $\kappa$, la \vocab{cofinalità} di $\kappa$, $\cof(\kappa)$, è il minimo cardinale 
	$\mu$ per cui esiste una famiglia $\{\lambda_i\}_{i \in \mu}$ di cardinali tali che:
	\[ \forall i \in \mu \; \lambda_i < \kappa \qquad \text e \qquad \kappa = \sum_{i \in \mu} \lambda_i
		\]
\end{definition}

\textcolor{MidnightBlue}{In altri termini, $\cof(\kappa)$ è il \textcolor{red}{minimo numero di parti} $<\kappa$ [ovvero proprio $\mu$] in cui può essere diviso un insieme di cardinalità $\kappa$.}

\begin{example}[Cofinalità v.1 di alcuni cardinali noti]
	Vediamone alcuni esempi pratici di cofinalità v.1, tenendo conto che cerchiamo sempre il minimo numero di ``pezzi'' in cui dividere un cardinale, in modo tale che i ``pezzi'' abbiano cardinalità strettamente minore:
	\begin{itemize}
		\item $\cof(\aleph_0) = \aleph_0$, qualsiasi cosa più piccola sarebbe un cardinale finito, e dividere $\aleph_0$ in un numero finito di parti dà ancora parti di cardinalità $\aleph_0$, mentre usando $\aleph_0$ abbiamo tutti ``pezzettini'' finiti, la cui unione finita è finita, ma l'unione di $\aleph_0$ cardinali finiti dà proprio $\aleph_0$ per le proprietà viste:
		\[ \aleph_0 = \sum_{i < \aleph_0} 1 \leq \sum_{i < \aleph_0} n_i \leq \sum_{i < \aleph_0} \aleph_0 = \aleph_0 \cdot \aleph_0 = \aleph_0
			\]
		\item $\cof(\aleph_\omega) = \aleph_0$ in quanto:
		\[ \sum_{\alpha < \aleph_0} \aleph_\alpha = \aleph_\omega
			\]
		e se usassimo $|I|<\aleph_0$ (ovvero un cardinale finito), accadrebbe che:
		\[\sum_{\alpha \in I} \aleph_\alpha = \max\left(|I|,\sup_{\alpha \in I} \aleph_\alpha\right) = \aleph_{\max(I)} < \aleph_\omega
		\]
		\item $\cof(\aleph_{42}) = \aleph_{42}$ in quanto:
		\[ \aleph_{42} = \sum_{i \in I} \kappa_i \leq \sum_{i \in I} \aleph_{41} = \max\left(|I|,\aleph_{41}\right) \implies |I| = \aleph_{42}
			\]
		dove la prima uguaglianza è il fatto che stiamo supponendo di poter scrivere $\aleph_{42}$ come somma, la seconda disuguaglianza deriva dal fatto che stiamo supponendo (per avere la definizione di cofinalità v.1) che i cardinali che sommiamo siano strettamente più piccoli di $\aleph_{42}$, e, nel caso peggiore (perché gli $\aleph_{41}$ sono i pezzi di
		grandezza massima che possiamo prendere), troviamo calcolando la somma, che 
		l'unica possibilità è che $|I| = \aleph_{42}$.
	\end{itemize}
\end{example}

\begin{definition}[Cofinalità - v.2]
	Dato un insieme ordinato $(S,<)$\footnote{Questa definizione di cofinalità, al contrario della precedente è valida su qualsiasi insieme [parzialmente] ordinato, mentre la
	precedente solo per i cardinali.}, diciamo che $A\subseteq S$ è \vocab{cofinale} in $S$ se $\forall x \in S \; \exists y \in A \; x \leq y$ \textcolor{MidnightBlue}{- ossia $A$ non 
	ha maggioranti stretti in S}. La \vocab{cofinalità} di $(S,<)$ è la minima cardinalità di un sottoinsieme cofinale di $S$.
\end{definition}

\begin{example}[Cofinalità v.2 di alcuni cardinali noti]
	Vediamone alcune secondo questa nuova definizione:
	\begin{itemize}
		\item $\cof(\omega) = \aleph_0$, $\omega$ è banalmente cofinale in se stesso, inoltre, qualsiasi altro sottoinsieme è finito, dunque ha un maggiorante stretto in $\omega$, pertanto $\omega$ è un\footnote{Va bene qualsiasi altro sottoinsieme infinito di $\omega$, tanto sono tutti $\aleph_0$.} sottoinsieme
		cofinale di $\omega$, da cui considerando la cardinalità si ha $\aleph_0$.
		\item $\cof(\RR,<) = \aleph_0$, infatti $\omega$ è cofinale in $\RR$, basta prendere per ogni $x \in \RR$ la parte intera superiore, $x \leq \lceil x \rceil \in \omega$, inoltre, per quanto visto con AC $\aleph_0$ è la più
		piccola cardinalità infinita (stiamo escludendo con lo stesso ragionamento di sopra che vi siano insiemi cofinali finiti in $\RR$), quindi per la definizione v.2, la minima cardinalità cercata è proprio $\aleph_0$.
		\item $\cof(]0,1],<_{|\RR}) = 1$, perché $\{1\}$ è un sottoinsieme cofinale di $]0,1]$ (non ha maggioranti stretti nel nostro intervallo), e la cofinalità non può essere 0 perché nell'intervallo ha maggioranti stretti.
		\item $\cof(\omega+1) = 1$, infatti, il sottoinsieme $\{\omega\} \subset \omega + 1$ non ha maggioranti stretti, per cui è cofinale ed in particolare $|\{\omega\}| = 1$, dunque, non essendo il vuoto (o 0) cofinale in $\omega + 1$, segue necessariamente che $\{\omega\}$ è il più piccolo sottoinsieme cofinale.
		\item $\cof(\omega_\omega) = \aleph_0$ perché $\{\omega_0,\omega_1,\omega_2,\ldots\} = \{\omega_\alpha | \alpha < \omega\}$ è un insieme cofinale in $\omega_\omega = \sup_{i \in \omega} \omega_\alpha$ [banalmente perché abbiamo preso tutti gli $\omega_\alpha$ prima di $\omega_\omega$, quindi non ci può essere fuori qualcosa di strettamente più grande], inoltre, la cardinalità di questo insieme è ovviamente $\aleph_0$,
		ed è il più piccolo insieme cofinale, infatti, per qualsiasi sottoinsieme finito di $\omega_\omega$, è sufficiente prendere il successivo dell'$\omega_\alpha$ più grande.
		\item $\cof(\omega_{42}) = \aleph_{42}$ per la proposizione che segue.
	\end{itemize}
\end{example}

\begin{proposition}[Equivalenza delle definizioni di cofinalità]
	Dato $\kappa$ cardinale \textbf{infinito}\footnote{Nel caso $\kappa$ cardinale finito, non coincidono, ed anzi, la definizione v.1 dà sempre 2, mentre la v.2 dà sempre 1, perché ogni cardinale finito ha sempre un massimo.},
	allora vale che $\cof^{\text{(v.1)}}(\kappa) = \cof^{\text{(v.2)}}(\kappa)$.
\end{proposition}

\begin{proof}
	Siano $\lambda_1 := \cof^{\text{(v.1)}}(\kappa)$ e $\lambda_2 := \cof^{\text{(v.2)}}(\kappa)$, verifichiamo quindi le due disuguaglianze per avere la tesi.
	\begin{itemize}
		\item[$\boxed{\lambda_1 \leq \lambda_2}$] Sia $A$ un sottoinsieme cofinale in $\kappa$, mostriamo che $\lambda_1 \leq |A|$, cioè che $\lambda_1$ è più piccolo di tutte le cardinalità dei sottoinsiemi cofinali di $\kappa$ in questo modo sarà proprio minore o uguale del minimo, ovvero $\lambda_2$.\\
		Osserviamo che (questo vale per tutti gli ordinali limite), essendo $\kappa$ un ordinale iniziale, dunque limite, $\forall \alpha \in \kappa \; \exists \beta \in A \; s(\alpha) \leq \beta$ (per cofinalità di $A$) e quindi $\alpha < \alpha + 1 \leq \beta$, dove $\alpha + 1 \in \kappa$ perché limite. Dunque ogni elemento di $\kappa$
		appartiene a qualche elemento di $A$, per cui vale $\kappa = \bigcup A$\footnote{Per essere precisi questo è un solo contenimento, l'altro segue dal fatto che $A \subseteq \kappa$ e $\kappa$ è un insieme transitivo, quindi $\bigcup A \subseteq \kappa$.}. Da ciò segue che:
		\[ \kappa \leq \sum_{\beta \in A} |\beta| \leq \sum_{\beta \in A} \kappa = |A| \cdot \kappa = \kappa
			\]
		per cui il termine in mezzo è uguale a $\kappa$, ed essendo che $\forall \beta \in A \; \beta \in \kappa \to |\beta| < \kappa$ (poiché $\kappa$ iniziale), quella ottenuta sopra è proprio una scrittura di $\kappa$ come somma di termini strettamente più piccoli, per cui si ha $ \lambda_1 \leq |A|$.
		\item[$\boxed{\lambda_2 \leq \lambda_1}$] Se $\lambda_1 = \kappa$ non c'è niente da dimostrare, perché $\kappa$ stesso è cofinale in sé, e quindi il minimo nella definizione v.2 comprende già $\kappa$ per cui la disuguaglianza è automaticamente verifica. Assumiamo quindi che $\lambda_1 < \kappa$, ci basta trovare un insieme cofinale in $\kappa$ di cardinalità $\lambda_1$, in questo modo sta nel minimo dell'altra definizione e otteniamo la disuguaglianza voluta.
		Per ipotesi esiste $\{\kappa_i\}_{i \in I}$ con $|I| = \lambda_1$, tale che per la definizione di cofinalità v.1:
		\[ \kappa = \sum_{i \in I} \kappa_i = \max\left(|I|,\sup_{i \in I} \kappa_i\right) = \max\left(\lambda_1, \sup_{i \in I} \kappa_i\right) \overset{\lambda_1 < \kappa}{=}\, \footnote{Se il max fosse $\lambda_1$, avremmo $\kappa = \lambda_1$, che è contro il fatto supposto poco sopra, ovvero $\lambda_1 < \kappa$, pertanto il max deve essere necessariamente il sup.} \sup_{i \in I} \kappa_i
			\]
		Osserviamo ora che $A := \{\kappa_i | i \in I\}$ è cofinale in $\kappa$ e da questo segue la tesi in quanto $\lambda_2 \overset{\text{def. cof$^{\text{(v.2)}}$}}{\leq} |A| \overset{i \twoheadrightarrow \kappa_i}{\leq} |I| = \lambda_1$.
		Dato $x \in \kappa$, se $x$ non fosse maggiorato da qualche elemento di $A$, avremmo che $x$ è un maggiorante di $A$ stesso, per cui:
		\[ \sup_{i \in I}\kappa_i \leq x < \kappa \; \lightning
			\]
		dove l'assurdo deriva dal fatto che sopra avevamo ottenuto che $\kappa = \sup_{i \in I}\kappa_i$. Alternativamente si può anche osservare che $x \in \kappa = \sup_{i \in I}\kappa_i = \bigcup_{i \in I}\kappa_i \left(= \bigcup A\right)$, ovvero $x \in \kappa_i$, per qualche $i \in I$, per cui $x < \kappa_i$, e dunque $A$ è cofinale in $\kappa$.
	\end{itemize}
\end{proof}

\begin{proposition}[Transitività della cofinalità]
	Sia $(S,<_S)$ totalmente ordinato e $T \subseteq S$ cofinale in $S$, allora:
	\[ \cof(S,<_S) = \cof(T,<_T)
		\]
	con $<_T = <_S \cap (T \times T)$.
\end{proposition}

\begin{proof}
	Per dimostrare la proposizione dimostriamo due cose, in primis che se $A \subseteq T$ è cofinale in $T$, allora è cofinale anche in $S$, dunque i sottoinsiemi cofinali di $T$ sono contenuti in quelli di $S$, per cui, per definizione v.2, si ha $\cof(S,<_S) \leq \cof(T,<_T)$.
	\begin{itemize}
		\item Dato $A \subseteq T$ cofinale in $T$, verifichiamo che è cofinale anche in $S$. Sia $x \in S$, dobbiamo trovare un $y \in A$ tale che $x \leq y$. Siccome $T$ è cofinale in $S$ per ipotesi, esiste $t \in T$ tale per cui $x \leq t$, e, siccome 
		$A$ è cofinale in $T$, esiste $y \in A$ per il quale $t \leq y$, da cui $x \leq y$.
	\end{itemize}
	La seconda cosa che dimostriamo è che se $B \subseteq S$ è cofinale in $(S,<_S)$, allora esiste un sottoinsieme $B' \subseteq T$, con $|B'| \leq |B|$, cofinale in $T$, in tal modo per la definizione v.2, si ha $\cof(T,<_T) \leq \cof(S,<_S)$, e si conclude la tesi.
	\begin{itemize}
		\item Dato $B \subseteq S$ cofinale, siccome $T$ è cofinale in $S$, possiamo usare la cofinalità di quest'ultimo rispetto a $B$, e quindi per ogni $b \in B \subseteq S$ possiamo fissare (in generale con AC) un $y_b \in T$ con $b \leq y_b$. Sia $B' := \{y_b \in T : b \in B\} \subseteq T$.\\
		Naturalmente la mappa $b \mapsto y_b$ è una funzione surgettiva da $B$ a $B'$, per cui abbiamo $|B'| \leq |B|$. Infine, non ci resta che osservare che $B'$ è cofinale in $T$, preso dunque $x \in T$, poiché $B$ è cofinale in $S$ esiste $b \in B \; b \geq x$,
		possiamo quindi considerare l'$y_b \in B'$ associato e ottenere $y_b \geq b \geq x$, che ci garantisce la cofinalità.
	\end{itemize}
\end{proof}

\begin{proposition}[I cardinali infiniti sono sempre cofinali in se stessi]
	Sia $\kappa$ un cardinale infinito, allora vale che $\cof(\kappa) \leq \kappa$.\footnote{Usando la definizione v.2, questo fatto è vero per un qualsiasi ordinale parziale.}
\end{proposition}

\begin{proof}
	Infatti, $\kappa$ si può scrivere come:
	\[ \kappa = \sum_{i \in \kappa} 1
		\]
	che rispetta la definizione v.1 di cofinalità. Pertanto $\kappa$ è tra le cardinalità di cui si prende il minimo nella definizione, e, essendo un elemento
	del minimo segue anche che necessariamente il minimo sara più piccolo, da cui $\cof(\kappa) \leq \kappa$.\\
	Ragionamento analogo se si usa la definizione v.2 di cofinalità, infatti ogni insieme è cofinale in se stesso, per cui $\kappa$ è tra gli insiemi di cui si considera il minimo della cardinalità per avere la cofinalità secondo questa definizione.
\end{proof}

\begin{remark}[La cofinalità è sempre un cardinale regolare]
	Sia $(S,<)$ totalmente ordinato, allora:
	\[ \cof(\cof(S,<)) = \cof(S,<)
		\]
	in particolare, per ogni cardinale infinito $\kappa$, si ha $\cof(\cof(\kappa)) = \cof(\kappa)$.
\end{remark}

\begin{proof}
	Sia $\kappa = \cof(S,<)$ e $A \subseteq S$ cofinale, quindi $|A| = \kappa$, sappiamo che vale in generale $\cof(\kappa) \leq \kappa$. Supponiamo per assurdo che $\cof(\kappa) < \kappa$,
	per definizione di cofinalità esistono $\{\kappa_i\}_{i \in I}$, $\forall i \in I \; \kappa_i < \kappa$ e $\cof(\kappa) = |I| < \kappa$, tali che:
	\[ \kappa = \sum_{i \in I} \kappa_i = |I| \cdot \sup_{i \in I}\kappa_i = \sup_{i \in I}\kappa_i
		\]
	dunque che possiamo fissare una famiglia $\{A_i\}_{i \in I}$ di sottoinsiemi di $A$ disgiunti tali che $\forall i \in I \; |A_i| = \kappa_i$ e $\bigcup\{A_i | i \in I\} = A$ (è l'osservazione vista all'inizio per cui data una famiglia arbitraria di insiemi disgiunti, la cardinalità della loro unione è proprio la somma dei cardinali associati).\\
	Osserviamo ora che ogni $i \in I$, $A_i$ \textcolor{red}{non è cofinale in $S$} siccome $|A_i| = \kappa_i < \kappa = \cof(S,<)$, quindi possiamo scegliere per tutti gli $A_i$ un maggiorante stretto $y_i \in S$.
	Sia $B := \{y_i \in S | i \in I\} \subseteq A$\footnote{Typo di Mamino.}, vale che $|B| \overset{i \twoheadrightarrow y_i}{\leq} |I| \overset{\text{Hp. assurda}}{<} \kappa$.\\
	D'altro canto $B$ è cofinale \textcolor{red}{in $S$}, infatti, preso $x \in S$, per la cofinalità di $A$,
	$\exists a \in A$ tale che $x \leq a$, ora $a$ è necessariamente in uno degli $A_i$, quindi si ha $x \leq a \leq y_i \in B$.
	Ma allora $B$ è cofinale in $(S,<)$ e ciò è assurdo perché $\cof(S,<) = \kappa$ e vale la disuguaglianza scritta sopra.
\end{proof}

Vediamo ora alcune applicazioni della cofinalità.

\begin{corollary}[di König - o disuguaglianze di cofinalità]
	Per $\kappa$ cardinale infinito, e $\lambda$ cardinale $\geq 2$ vale:
	\[ \kappa < \kappa^{\cof(\kappa)} \qquad \kappa < \cof(\lambda^{\kappa})\,
		\]
\end{corollary}

\begin{proof}
	Cominciamo dalla prima disuguaglianza. Per definizione di $\cof(\kappa)$ esiste una famiglia $\{\kappa_i\}_{i \in \cof(\kappa)}$, con $\kappa_i < \kappa$ per ogni $i \in \cof(\kappa)$, dunque usando König si ha:
	\[ \kappa = \sum_{i \in \cof(\kappa)} \kappa_i \; \textcolor{red}{<} \; \prod_{i \in \cof(\kappa)} \kappa = \kappa^{\cof(\kappa)}
		\]
	Per la seconda disuguaglianza ragioniamo per assurdo e assumiamo che $\kappa \geq \cof(\lambda^\kappa)$. A questo punto possiamo applicare la disuguaglianza precedente a $\lambda^\kappa$ ed ottenere:
	\[ \lambda^\kappa \;\textcolor{red}{<}\; (\lambda^\kappa)^{\cof(\lambda^\kappa)} \overset{\text{Hp. assurda}}{\leq} (\lambda^{\kappa})^{\kappa} = \lambda^{\kappa \cdot \kappa} = \lambda^{\kappa} \;\lightning
		\]
\end{proof}

\begin{definition}[Funzione $\gimel$]
	La funzione $\gimel$\footnote{\;``Gimel'', pronunciata ``ghimel''.} è definita sui cardinali infiniti da:
	\[ \gimel(\kappa) = \kappa^{\cof(\kappa)}
		\]
\end{definition}

\begin{corollary}[Disuguaglianze di cofinalità via $\gimel$]
	Con la funzione $\gimel$, dato $\kappa$ cardinale infinito, abbiamo le disuguaglianze di cofinalità:
	\[ \kappa < \gimel(\kappa) \qquad \cof(\kappa) < \cof(\gimel(\kappa))
		\]
\end{corollary}

\begin{proof}
	La prima disuguaglianza è una riscrittura di quella sopra, infatti $\kappa < \kappa^{\cof(\kappa)} = \gimel(\kappa)$. La seconda disuguaglianza è un corollario della seconda disuguaglianza sopra che
	si ottiene prendendo $\lambda = \kappa$ ed usando $\cof(\kappa)$ al posto di $\kappa$.
\end{proof}

\begin{definition}[Proprietà dei cardinali]
	Dato un cardinale $\kappa$, diciamo che è:
	\begin{itemize}
		\item \vocab{regolare} se $\cof(\kappa) = \kappa$
		\item \vocab{singolare} se $\cof(\kappa) < \kappa$
		\item \vocab{successore} se $\kappa = \aleph_{\alpha + 1}$ per qualche $\alpha \in \Ord$
		\item \vocab{limite} se $\kappa = \aleph_{\lambda}$ per $\lambda$ ordinale limite
		\item \vocab{limite forte} se $\aleph_0 < \kappa$ e $\forall \alpha \in \Ord \; \aleph_\alpha < \kappa \rightarrow 2^{\aleph_\alpha} < \kappa$
		\item \vocab{debolmente inaccessibile} se è regolare e limite
		\item \vocab{(fortemente) inaccessibile} se è regolare e limite forte.
	\end{itemize}
\end{definition}

\begin{proposition}[Limite forte $\implies$ limite]
	Un cardinale limite forte è anche limite.
\end{proposition}

\begin{proof}
	Se per assurdo fosse successore, $\kappa = \aleph_{\alpha + 1}$, poiché $\aleph_\alpha < \aleph_{\alpha + 1}$ avremmo per ipotesi di limite forte che $2^{\aleph_\alpha} < \aleph_{\alpha + 1}$, ma questo è assurdo in quanto $\aleph_{\alpha + 1} \leq 2^{\aleph_\alpha}$ poiché è per definizione il più piccolo cardinale strettamente maggiore di $\aleph_{\alpha}$.
\end{proof}

\begin{proposition}[Successore $\implies$ regolare]
	I cardinali successori sono regolari.
\end{proposition}

\begin{proof}
	Sia $\kappa = \aleph_{\alpha + 1}$, da sopra sappiamo che vale sempre $\cof(\kappa) \leq \kappa$, supponiamo per assurdo che $\kappa$ non sia regolare, cioè $\cof(\kappa) < \kappa$. Usando la definizione di cofinalità si ottiene:
	\[ \aleph_{\alpha + 1} = \kappa = \sum_{i \in \cof(\kappa)} \kappa_i \leq \sum_{i \in \cof(\kappa)} \aleph_\alpha = \cof(\kappa) \cdot \aleph_\alpha \leq \aleph_\alpha \;\textcolor{red}{<}\; \aleph_{\alpha + 1} \; \lightning
		\]
	dove nella penultima disuguaglianza abbiamo usato che $\cof(\kappa) < \kappa = \aleph_{\alpha + 1} \implies \cof(\kappa) \leq \aleph_\alpha$, poiché $\aleph_\alpha$ è il più grande cardinale strettamente più piccolo di $\aleph_{\alpha + 1}$.
\end{proof}

\begin{corollary}[$\gimel$ di cardinali regolari]
	Se $\kappa$ è \textcolor{red}{successore}, allora $\gimel(\kappa) = 2^{\kappa}$.
\end{corollary}

\begin{proof}
	È ovvio che $\gimel(\kappa) = \kappa^{\cof(\kappa)} \overset{\text{$\kappa$ regolare}}{=} \kappa^\kappa = 2^{\kappa}$, dove l'ultima disuguaglianza come al solito vale perché:
	\[ 2^{\kappa} \overset{\textcolor{purple}{2 \leq \kappa}}{\leq} \kappa^{\kappa} \overset{\textcolor{purple}{\kappa \leq 2^\kappa}}{\leq} (2^{\kappa})^{\kappa} = 2^{\kappa}
		\]
\end{proof}

\begin{proposition}[Cofinalità dei cardinali limiti]
	Se $\lambda$ è un cardinale limite, allora $\cof(\aleph_\lambda) = \cof(\lambda)$.
\end{proposition}

\begin{proof}
	Osserviamo che $\{\aleph_\alpha : \alpha < \lambda\}$ è un sottoinsieme cofinale in $\aleph_\lambda$, infatti, dato $y \in \aleph_\lambda = \bigcup_{\alpha < \lambda} \aleph_\alpha$, si ha $y \in \aleph_\alpha$, per qualche $\alpha < \lambda$, dunque è sempre maggiorato da un elemento dell'insieme.
	Inoltre, essendo la mappa $\alpha \mapsto \aleph_\alpha$ strettamente crescente e surgettiva si ha $\{\aleph_\alpha : \alpha < \lambda\} \sim \lambda$\footnote{Stiamo implicitamente usando che la cofinalità di insiemi totalmente ordinati isomorfi sia la stessa, questa proprietà è un'immediata conseguenza del fatto di star usando una mappa 
	strettamente crescente tra gli insiemi, da cui si ottiene che se le cofinalità non fossero uguali si avrebbe un assurdo.}, dunque la cofinalità di questo insieme è proprio uguale a $\cof(\lambda)$, e si conclude ricordando che abbiamo visto che la cofinalità è transitiva per i sottoinsiemi cofinali, dunque $\cof(\aleph_\lambda) = \cof(\lambda)$.
\end{proof}
\pagebreak
\begin{note}[Esistenza di cardinali inaccessibili nella ZFC]
	In generale, un cardinale \textcolor{red}{limite} potrebbe essere \textcolor{MidnightBlue}{singolare o regolare}. Tuttavia, benché sia facile esibire cardinali limiti singolari (abbiamo visto ad esempio $\aleph_\omega$), gli assiomi della ZFC non implicano l'esistenza [all'interno della teoria]
	di un cardinale limite e regolare - che abbiamo detto si chiama appunto \textcolor{purple}{debolmente inaccessibile}. Insomma, non si può dimostrare che tutti i cardinali limiti sono singolari - e non c'è una buona ragione per assumere che lo siano (quindi non aggiungiamo un altro assioma) - ma sarebbe comunque coerente
	con gli assiomi della teoria degli insiemi che tutti i cardinali limiti fossero singolari (quindi potremmo aggiungere questo assioma senza perdere la coerenza).
\end{note}

\begin{remark}[Cardinali limiti ed inaccessibili]
	L'idea dietro ai cardinali debolmente limiti è che non possano essere ``raggiunti'' a partire da un altro cardinale ripetendo l'operazione di successore cardinale (proprio come accade per gli ordinali limite con l'operazione di successore ordinale).\\
	I cardinali che sono limiti forti, oltre alla proprietà precedente, per definizione, non sono ottenibili ripetendo l'operazione di prendere le parti\footnote{Un esempio può essere $\beth_\omega$, che è un cardinale limite forte [singolare], che, da definizione (si veda negli esercizi in fondo), non è ottenibile prendendo le parti di un qualche $\beth_\alpha$, per $\alpha < \omega$.},
	infatti vale appunto che se $\lambda$ è limite forte, preso $\kappa < \lambda$ si ha $2^\kappa < \lambda$, dunque prendere le parti di cardinali più piccoli non ci fa arrivare mai a $\kappa$.\\
	I cardinali limiti e limiti forti possono essere ancora costruiti per unione (come ad esempio capita per i $\beth_\lambda$, con $\lambda$ limite), quindi avere un cardinale limite forte non è ancora sufficiente per avere qualcosa di inaccessibile, la nozione vera di inaccessibilità può essere definita solo tramite la cofinalità. Infatti,
	per un cardinale $\kappa$ limite [forte] richiedere che sia regolare ci dice che tale cardinale \textcolor{purple}{non} può essere espresso come somma (unione) di meno di $\kappa$ cardinali minori di $\kappa$.
\end{remark}

\pagebreak
\subsection{Formula di Hausdorff}

\begin{proposition}[Formula di Hausdorff]
	Siano $\kappa$ e $\lambda$ cardinali infiniti, allora:
	\[ (\kappa^+)^\lambda = \kappa^{\lambda} \cdot \kappa^+
		\]
\end{proposition}

\begin{proof}
	Dimostriamo le due disuguaglianze tra LHS e RHS.
	\begin{itemize}
		\item[$\boxed{\geq}$] Siccome $\kappa^{\lambda} \cdot \kappa^+ = \max(\kappa^{\lambda}, \kappa^+)$, è ovvio che $(\kappa^+)^\lambda$ è maggiore o uguale di entrambi i termini per debole monotonia dell'esponenziale. Alternativamente, sempre sfruttando la debole monotonia dell'esponenziale si ha:
		\[ \kappa^\lambda \cdot \kappa^+ \leq (\kappa^+)^{\lambda} \cdot \kappa^+ = (\kappa^+)^{\lambda + 1} \overset{\text{$\lambda$ infinito}}{=}(\kappa^+)^{\lambda}
			\]
		\item[$\boxed{\leq}$] Distinguiamo due casi:
		\begin{itemize}
			\item[$\bullet$]\textbf{\underline{se $\kappa < \lambda$}}: in questo caso si ha, per definizione, che $\kappa^+ \leq \lambda$, da cui:
			\[ (\kappa^+)^\lambda \leq \lambda^\lambda = 2^{\lambda} \overset{\textcolor{purple}{\kappa < \lambda}}{=} \kappa^{\lambda} \leq \kappa^{\lambda}\cdot \kappa^+
				\]
			\item[$\bullet$]\textbf{\underline{se $\kappa \geq \lambda$}}: essendo $\kappa^+$ successore, è regolare, dunque $\cof(\kappa^+) = \kappa^+$, si osserva dunque che data $f \in (\kappa^+)^\lambda$ si ha:
			\[ |\Imm(f)| \leq \lambda \leq \kappa < \kappa^{+} \implies \text{$\Imm(f)$ non cofinale in $\kappa^+$}
				\]
			ovvero esiste un ordinale $\alpha \in \kappa^+$ tale che [è un maggiorante di tutta l'immagine] $\Imm(f) \subseteq \alpha$, per cui segue che $f \in \alpha^\lambda$. Quanto visto implica che $(\kappa^+)^\lambda \subseteq \bigcup_{\alpha < \kappa^+}(\alpha)^\lambda$, ed essendo che 
			per ogni $\alpha \in \kappa^+$ è ovvio che $\alpha^{\lambda} \subseteq (\kappa^+)^\lambda$, si ha anche che $\bigcup_{\alpha < \kappa^+} \alpha^\lambda \subseteq (\kappa^+)^\lambda$.\\
			A questo punto possiamo stimare la cardinalità di $(\kappa^+)^\lambda$ facendo uso della disuguaglianza di inclusione-esclusione come segue:
			\[ (\kappa^+)^\lambda = \left\lvert \bigcup_{\alpha < \kappa^+} \alpha^{\lambda}\right\rvert \leq \sum_{\alpha < \kappa^+} |\alpha^{\lambda}| = \sum_{\alpha < \kappa^+} |\alpha|^{\lambda} \leq \sum_{\alpha < \kappa^+} \kappa^{\lambda} = \kappa^{\lambda} \cdot \kappa^+
				\]
		\end{itemize}
	\end{itemize}
\end{proof}

\begin{remark}
	[Alternativa per la seconda disuguaglianza]
	Nella dimostrazione precedente, per dimostrare che $(\kappa^+)^\lambda \leq \kappa^\lambda \cdot \kappa^+$, nel caso in cui $\kappa < \lambda$, si poteva procedere alternativamente, sfruttando il fatto che $\kappa < 2^\kappa \to \kappa^+ \leq 2^k$:
	\[ (\kappa^+)^\lambda \leq (2^{\kappa})^{\lambda} = (\kappa^{\lambda})^{\lambda} = \kappa^\lambda \leq \kappa^\lambda \cdot \kappa^+
		\]	
\end{remark}

\pagebreak

\begin{remark}
	[Disuguaglianza somma-prodotto]
	Data una famiglia infinita di cardinali $\{\kappa_i\}_{i \in I}$ vale una disuguaglianza somma-prodotto\footnote{König normalmente ha come condizione iniziale una disuguaglianza
	stretta che ne implica un'altra stretta, e quest'ultima a sua volta ne implica una larga, tuttavia, questo ragionamento non funziona se la disuguaglianza iniziale è larga perché non potremmo applicare König.}:
	\[ \sum_{i \in I} \kappa_i \leq \prod_{i \in I} \kappa_i
		\]
\end{remark}

\begin{proof}
	Ci basta trovare una funzione iniettiva:
	\[  F : \bigcup_{i \in I} \kappa_i \times \{i\} \to \prod_{i \in I}\kappa_i = \left\{f : I \to \sup_{i \in I}\kappa_i \m \forall i \in I \; f(i) \in \kappa_i\right\}
		\]
	Osserviamo che essendo il prodotto fatto da cardinali infiniti, dunque aleph, vale che $\prod_{i \in I} \kappa_i = \prod_{i \in I} (\kappa_i + 1)$\footnote{Stiamo aggiungendo a tutti i cardinali $\{\spadesuit\}$ e fare il conto così non cambia, perché abbiamo visto che il prodotto è ben definito (cioè anche usando un insieme che non sia il cardinale stesso, purché abbia la stessa cardinalità, ciò che esce dal prodotto è in bigezione col risultato del prodotto fatto con i cardinali).}, con quest'ultimo che è dato dall'insieme:
	\[ \left\{f : I \to \sup_{i \in I}(\kappa_i + 1) \m \forall i \in I \; f(i) \in \kappa_i \cup \{\spadesuit\}\right\}
		\]
	A questo punto la mappa più naturale possibile associa ogni coppia $(a_i,i)$ (con $a_i \in \kappa_i$) ad una funzione nell'insieme di arrivo che su $i$ fa $a_i$ e fa $\spadesuit$ altrove:
	\[ F : \bigcup_{i \in I} \kappa_i \hookrightarrow \prod_{i \in I} (\kappa_i + 1) : (a_i,i) \mapsto \underbrace{f(x)}_{\in {}^I(\sup_{i \in I}(\kappa_i + 1))} = \begin{cases}
		a_i &\text{se $x = i$} \\
		\spadesuit &\text{se $x \ne i$}
	\end{cases}
		\]
	ed è iniettiva appunto perché la coppia $(a_i,i)$ può essere pensata mappata nella $|I|$-upla (o funzione) $(\spadesuit,\ldots,\underbrace{a_i}_{i},\ldots,\spadesuit,\ldots)$, dunque è facile vedere che due funzioni uguali ci danno che le coppie in partenza sono esattamente uguali.
\end{proof}

\begin{exercise}[$\aleph_\omega^{\aleph_0}$]
	Dimostrare che $\aleph_\omega^{\aleph_0} = \prod_{n \in \omega} \aleph_n$.
\end{exercise}

\begin{soln}
	Vediamo le due disuguaglianze.
	\begin{itemize}
		\item[$\boxed{\geq}$] Facile stima che deriva dalla monotonia del prodotto cardinale:
		\[ \prod_{n \in \omega} \aleph_n \leq \prod_{n \in \omega} \aleph_\omega = \aleph_\omega^{\aleph_0}
			\]
		\item[$\boxed{\leq}$] Consideriamo $\Lambda = \bigsqcup \{\Lambda_k | k \in \omega\} \subseteq \omega$, con $\Lambda_k = \{p_k^{n+1} | n \in \omega\}$ (abbiamo escluso 1 da tutti gli insiemi per renderli [insiemi numerabili] disgiunti), dove $p_k$ è il $k$-esimo primo (ricordiamo che sono enumerabili da $\omega$).\\
		A questo punto abbiamo:
		\begin{align*}
			\prod_{n \in \omega} \aleph_n &\geq \prod_{n \in \Lambda} \aleph_n &&\text{($\Lambda \subseteq \omega$)} \\
			&= \prod_{n \in \Lambda_1 \sqcup \ldots \sqcup \Lambda_k \sqcup \ldots} \aleph_n \\
			&= \left(\prod_{n \in \Lambda_1}\aleph_n\right)\cdot\ldots\cdot\left(\prod_{n \in \Lambda_k}\aleph_n\right)\cdot\ldots &&\text{(associatività infinita)} \\
			&=\prod_{k \in \omega}\prod_{n \in \Lambda_k}\aleph_n \\
			&\geq \prod_{k \in \omega}\left(\sum_{n \in \Lambda_k}\aleph_n\right) &&\text{(disuguaglianza somma-prodotto)} \\
			&= \prod_{k \in \omega}\left(|\aleph_0| \cdot \sup_{n \in \omega}\aleph_n\right) \\
			&= \prod_{k \in \omega}\aleph_\omega = \aleph_\omega^{\aleph_0}
		\end{align*}
	\end{itemize}
\end{soln}

\begin{remark}[Alternativa per la seconda disuguaglianza precedente]
	Osserviamo che al posto di usare l'associatività infinita su un'unione disgiunta infinita di insiemi, nella seconda disuguaglianza precedente, avremmo anche potuto fissare una bigezione $f : \omega \times \omega \to \omega$
	e riscrivere la produttoria come segue [cambiando di fatto solo il nome dell'insieme su cui la facciamo]:
	\[ \prod_{n \in \omega} \aleph_n = \prod_{(a,b)\in \omega \times \omega} \aleph_{f(a,b)} = \prod_{a \in \omega} \prod_{b \in \omega} \aleph_{f(a,b)}
		\]
	(dove l'ultima uguaglianza è semplicemente una riscrittura dell'insieme), osserviamo ora che il prodotto è maggiore di tutti gli elementi di $\{\aleph_{f(a,b)} | b \in \omega\}$ dunque è maggiore o uguale del sup di tale insieme (per definizione):
	\[ \prod_{b \in \omega} \aleph_{f(a,b)} \geq \sup\{\aleph_{f(a,b)} | b \in \omega\} = \aleph_\omega
		\]
	dove l'uguaglianza deriva dal fatto che l'insieme su cui stiamo facendo il sup coincide con $\{\aleph_n | n \in \omega\}$. Dunque possiamo stimare $\prod_{b \in \omega} \aleph_{f(a,b)} \geq \aleph_\omega$ e ottenere:
	\[ \prod_{n \in \omega} \aleph_n \geq \prod_{a \in \omega} \aleph_\omega = \aleph_\omega^{\aleph_0}
		\]
\end{remark}

\begin{exercise}[$\aleph_n^{\aleph_1} = \aleph_0^{\aleph_1}\cdot \aleph_n$]
	Dimostrare che per ogni $n \in \omega$ vale $\aleph_n^{\aleph_1} = \aleph_0^{\aleph_1}\cdot \aleph_n$.\footnote{Osservare che con la stessa identica dimostrazione vale di più: \textcolor{purple}{$\aleph_n^{\aleph_\alpha} = \aleph_0^{\aleph_\alpha} \cdot \aleph_n$}, per $\alpha \in \Ord$.}
\end{exercise}

\begin{soln}
	Procediamo per induzione numerabile.
	\begin{itemize}
		\item[$\boxed{\text{caso $n = 0$}}$] In questo caso si ha che $\aleph_0^{\aleph_1} = \aleph_0^{\aleph_1} \cdot \aleph_0$, infatti per monotonia dell'esponenziale $\aleph_0 \leq \aleph_0^{\aleph_1}$, dunque nel prodotto tra cardinali $\aleph_0$ viene assorbito.
		\item[$\boxed{\text{caso $n + 1$}}$] Supponiamo per ipotesi induttiva che $\aleph_n^{\aleph_1} = \aleph_0^{\aleph_1}\cdot \aleph_n$ e osserviamo che:
		\[ \aleph_{n+1}^{\aleph_0} \overset{\text{Hausdorff}}{=} \aleph_n^{\aleph_0} \cdot \aleph_{n+1} \overset{\text{Hp. induttiva}}{=} \aleph_0^{\aleph_1}\cdot \aleph_n \cdot \aleph_{n+1} = \aleph_0^{\aleph_1} \cdot \aleph_{n+1}
			\]
	\end{itemize}
\end{soln}

\begin{exercise}[$\aleph_\omega^{\aleph_1}$]
	Dimostrare che $\aleph_\omega^{\aleph_1} = \gimel(\aleph_1) \cdot \gimel(\aleph_\omega)$.\footnote{\underline{\textbf{Hint}}: Studiare $\left(\prod_{n \in \omega} \aleph_n\right)^{\aleph_1}$.}
\end{exercise}

\begin{soln}
	Osserviamo innanzitutto che $\gimel(\aleph_1) = \aleph_1^{\cof(\aleph_1)} = \aleph_1^{\aleph_1}$ (abbiamo appena visto che i cardinali successori sono regolari) e $\gimel(\aleph_\omega) = \aleph_{\omega}^{\cof(\aleph_{\omega})} = \aleph_{\omega}^{\cof(\omega,<)} = \aleph_\omega^{\aleph_0}$,
	quindi il RHS è $\aleph_1^{\aleph_1} \cdot \aleph_\omega^{\aleph_0}$. Si tratta quindi di dimostrare come al solito due disuguaglianze.
	\begin{itemize}
		\item[$\boxed{\geq}$] È una facile stima sfruttando la monotonia dell'esponenziale e le proprietà delle potenze:
		\[ \aleph_1^{\aleph_1}\cdot \aleph_\omega^{\aleph_0} \leq \aleph_1^{\aleph_1}\cdot \aleph_\omega^{\aleph_1} = (\aleph_1 \cdot \aleph_\omega)^{\aleph_1} = \aleph_\omega^{\aleph_1}
			\] 
		\item[$\boxed{\leq}$] Per questa disuguaglianza sono necessari i due esercizi precedenti, infatti:
		\begin{align*}
			\aleph_\omega^{\aleph_1} &= \left(\sum_{n \in \omega} \aleph_n\right)^{\aleph_1} \\
									 &\leq \left(\prod_{n \in \omega} \aleph_n\right)^{\aleph_1} &&\text{(disuguaglianza somma-prodotto)} \\
									 &= \prod_{n \in \omega}\aleph_n^{\aleph_1} \\
									 &= \prod_{n \in \omega} \aleph_0^{\aleph_1} \cdot \aleph_n &&\text{($\aleph_n^{\aleph_1} = \aleph_0^{\aleph_1}\cdot\aleph_n$)} \\
									 &= \aleph_0^{\aleph_1} \cdot \prod_{n \in \omega} \aleph_n \\
									 &= \aleph_0^{\aleph_1} \cdot \aleph_\omega^{\aleph_0} \leq \aleph_1^{\aleph_1} \cdot \aleph_\omega^{\aleph_0}
		\end{align*}
	\end{itemize}
\end{soln}

\begin{fact}[Esponenziazione di cardinali]
	La funzione esponenziale $\kappa^\lambda$ è determinata ricorsivamente dalle funzioni $\cof$ e $\gimel$.
\end{fact}

Non dimostriamo questo fatto, dimostriamo tuttavia il seguente caso particolare, che basta ad illustrare le tecniche necessarie: la \vocab{funzione del continuo}
$\kappa \mapsto 2^{\kappa}$ è determinata ricorsivamente dalla funzione $\gimel(\kappa)$.

\begin{definition}[Quasi esponenziazione di un cardinale limite]
	Sia $\kappa$ un cardinale \textcolor{red}{limite}. Definiamo:
	\[ 2^{<\kappa} \Mydef \sup\{2^\lambda | \; \text{$\lambda$ \textcolor{red}{cardinale} $<\kappa$}\}
		\]
\end{definition}

\begin{lemma}[2 alla cardinle limite]
	Dato $\kappa$ cardinale limite vale:
	\[ 2^{\kappa} = (2^{<\kappa})^{\cof(\kappa)}
		\]
\end{lemma}

\begin{proof}
	Per definizione di cofinalità sia $\{\kappa_i\}_{i \in \lambda}$, tale che $\kappa_i < \kappa$ per ogni $i \in \cof(\kappa)$, per cui $\kappa = \sum_{i \in \cof(\kappa)}\kappa_i$. Allora vale la segue catena di disuguaglianze:
	\[  \textcolor{purple}{2^\kappa} = 2^{\sum_{i \in \cof(\kappa)}\kappa_i}
									 = \prod_{i \in \cof(\kappa)} 2^{<\kappa}
									 = \textcolor{MidnightBlue}{(2^{<\kappa})^{\cof(\kappa)}}
									 \leq (2^\kappa)^{\cof(\kappa)} = \textcolor{purple}{2^\kappa}
		\]
	dove: la prima uguaglianza è la definizione di cofinalità, la seconda la proprietà del prodotto infinito di potenze (visto in un'osservazione ad inizio capitolo), la terza è una delle proprietà dei prodotti infiniti,
	e la quarta è la monotonia debole dell'esponenziale, in quanto $2^{<\kappa} = \sup\{2^{\lambda}|\text{$\lambda$ cardinale e } \lambda < \kappa\} \leq 2^\kappa$.
\end{proof}

\begin{definition}[Funzione del continuo definitivamente costante sotto un cardinale]
	Sia $\kappa$ un cardinale, diciamo che la funzione del continuo $\lambda \mapsto 2^\lambda$ è \vocab{definitivamente costante sotto $\kappa$} se esiste un cardinale $\mu < \kappa$ tale che [la funzione del
	continuo è costante da lì in poi fino a $\kappa$ escluso] $\forall \nu$ cardinale $\mu \leq \nu < \kappa \rightarrow 2^\nu = 2^\mu$.
\end{definition}

\begin{proposition}[Esponenziazione cardinale di 2]
	La funzione del continuo $\kappa \mapsto 2^{\kappa}$ è determinata da $\gimel$ come segue:
	\[ 2^{\kappa} = \begin{cases}
		\gimel(\kappa) &\text{se $\kappa$ è successore} \\
		2^{<\kappa} \cdot \gimel(\kappa) &\text{se $\lambda \mapsto 2^\lambda$ è definitivamente costante sotto $\kappa$} \\
		\gimel(2^{<\kappa}) &\text{se $\lambda \mapsto 2^\lambda$ \textcolor{red}{non} è definitivamente costante sotto $\kappa$} \\
	\end{cases}
		\]
\end{proposition}

\begin{proof}
	Abbiamo già visto il caso successore in precedenza. Supponiamo quindi che $\kappa$ sia un cardinale limite e vediamo prima i casi in cui 
	la funzione del continuo è definitivamente costante e $\kappa$.
	\begin{itemize}
		\item \textcolor{purple}{Caso $\lambda \mapsto 2^\lambda$ è definitivamente costante sotto $\kappa$ e $\kappa$ regolare.}\\
		Osserviamo che vale:
		\[ 2^{<\kappa} \cdot \gimel(\kappa) \overset{\text{$\kappa$ regolare}}{=} 2^{<\kappa} \cdot \kappa^\kappa = 2^{<\kappa} \cdot 2^\kappa = 2^\kappa
			\]
		dove nell'ultima uguaglianza abbiamo usato che $2^{\kappa}$ è un maggiorante dell'insieme $\{2^{\lambda} | \;\text{$\lambda < \kappa$ e $\lambda$ cardinale}\}$, dunque è maggiore o uguale del sup, ovvero $2^{<\kappa}$.
		\item \textcolor{purple}{Caso $\lambda \mapsto 2^\lambda$ è definitivamente costante sotto $\kappa$ e $\kappa$ singolare.}\\
		In questo caso l'ipotesi che la funzione del continuo sia definitivamente costante sotto $\kappa$ ci dice che $\exists \mu < \kappa$ tale che per ogni cardinale $\mu \leq \nu < \kappa$, si ha $2^{\mu} = 2^{\nu}$, in particolare ciò significa che $2^{\mu} = 2^{<\kappa}$ (soddisfa esattamente la definizione di sup 
		dell'insieme), e ciò ci dà:
		\[ \textcolor{orange}{2^\kappa} \overset{\text{lemma}}{=} (2^{<\kappa})^{\cof(\kappa)} = 2^{\overbrace{\mu \cdot \cof(\kappa)}^{\mu \leq \;< \kappa}} = 2^\mu = \textcolor{MidnightBlue}{2^{<\kappa}} 
			\]
		Per concludere osserviamo che in questo caso $\gimel(\kappa) = 2^{<\kappa}$:
		\[ \textcolor{MidnightBlue}{\gimel(\kappa)} = {\kappa}^{\cof(\kappa)} \overset{\textcolor{red}{\cof(\kappa) < \kappa}}{\leq} \kappa^{\kappa} = 2^\kappa \overset{\text{sopra}}{=} \textcolor{MidnightBlue}{2^{<\kappa}}
			\]
	\end{itemize}
	Vediamo ora il caso in cui la funzione del continuo non è definitivamente costante sotto $\kappa$.
	\begin{itemize}
		\item \textcolor{purple}{Caso $\lambda \mapsto 2^\lambda$ non definitivamente costante sotto $\kappa$.}\\
		Basta dimostrare che \textcolor{green}{$\cof(2^{<\kappa}) = \cof(\kappa)$}, e da questo segue immediatamente che:
		\[ \textcolor{orange}{2^\kappa} = (2^{<\kappa})^{\cof(\kappa)}\; \textcolor{green}{=}\; (2^{<\kappa})^{\cof(2^{<\kappa})} = \textcolor{MidnightBlue}{\gimel(2^{<\kappa})}
			\]
		dunque non dobbiamo far altro che dimostrare l'assunto.
		\begin{itemize}
			\item[$\boxed{\cof(2^{<\kappa}) \leq \cof(\kappa)}$] Sia $A \subseteq \kappa$ cofinale in $\kappa$ e consideriamo $B:=\{2^{\alpha} | \alpha \in A\}\subseteq 2^{<\kappa}$, naturalmente si ha $|B| \overset{\alpha \twoheadrightarrow 2^{\alpha}}{\leq} |A|$, se verifichiamo che $B$ è cofinale in $2^{<\kappa}$, avremmo la disuguaglianza per la definizione v.2 di cofinalità.\\
			Sia $x < 2^{<\kappa}$, allora, per definizione di sup, esiste $\gamma < \kappa$ tale che $x < 2^\gamma$, siccome $A$ è cofinale in $\kappa$, allora esiste $\beta \in A$ tale che $\beta \geq \gamma > x$, dunque per monotonia si ottiene $x < 2^\gamma \leq 2^\beta \in B$.
			\item[$\boxed{\cof(\kappa) \leq \cof(2^{<\kappa})}$] Per definizione $\alpha < 2^{<\kappa}$ significa che esiste un cardinale $\beta < \kappa$, tale che $\alpha < 2^\beta$, indichiamo con $\beta_\alpha$ il minimo cardinale per cui ciò accade.
			Sia ora $A \subseteq 2^{<\kappa}$ un sottoinsieme cofinale, definiamo $B:=\{\beta_\alpha : \alpha \in A\}$, naturalmente la mappa $\alpha \mapsto \beta_\alpha$ ci dà la disuguaglianza $|B| \leq |A|$, se dimostriamo quindi che $B$ è cofinale in $\kappa$ abbiamo concluso.\\
			Sia $x < \kappa$, si ha $2^x \leq 2^\kappa$, non può valere che $2^x = 2^\kappa$, altrimenti la funzione del continuo sarebbe definitivamente costante sotto $\kappa$ (contro l'ipotesi), per cui esiste $x < y < \kappa$ tale che $2^x < 2^y < 2^\kappa$. Quanto detto ci assicura che $2^x < 2^\kappa$,
			a questo punto, per la cofinalità di $A$, si ha $2^x \leq \alpha \in A$, da cui $2^x \leq \alpha < 2^{\beta_\alpha}$. Osserviamo infine che $x \leq \beta_\alpha$, infatti, se fosse $x > \beta_\alpha$, allora $2^x \geq 2^{\beta_\alpha}$, che è assurdo.
		\end{itemize}
	\end{itemize}
\end{proof}