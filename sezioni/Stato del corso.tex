\section*{Stato del corso}
\addcontentsline{toc}{section}{Stato del corso}
È un dato di fatto - il primo teorema di incompletezza di Gödel - che ogni teoria \vocab{calcolabile} - i cui assiomi possano, cioè, essere elencati 
in maniera meccanica - è necessariamente incompleta. L'incompletezza non è quindi un difetto, o meglio, che lo sia oppure no è irrilevante, perché 
non può essere evitata.\\
Tuttavia, gli assiomi che abbiamo introdotto fino ad ora lasciano aperte lacune che sarebbe desiderabile colmare.

\begin{enumerate}[1.]
	\item Sarebbe ragionevole che questi insiemi esistessero [all'interno della teoria che stiamo costruendo]:
	\[ \{\emptyset, \{\emptyset\}, \{\{\emptyset\}\}, \{\{\{\emptyset\}\}\}, \ldots\}
		\]\[ \{\omega, s(\omega), s(s(\omega)), s(s(s(\omega))),\ldots\}
			\]
	Però gli assiomi 1-7 non bastano né per dimostrarne l'esistenza, né - e questo sarebbe disastroso - permettono di escluderla.
	\item Alcune questioni sulle cardinalità, come per esempio la confrontabilità, non possono essere decise sulla base dei solo assiomi 1-7.
	Inoltre ci mancano risultati desiderabili per via delle applicazioni, segnatamente il lemma di Zorn.
	\item Vi sono insiemi la cui esistenza vorremmo escludere. Per esempio vorremmo che l'equazione $X = \{X\}$ non avesse soluzioni, e farebbe comodo escludere 
	l'esistenza di qualcosa del tipo $Y = \{\{\{\{\{\ldots\}\}\}\}\}$ con infinite parentesi annidate. Il guaio qui non è grave, ma questi oggetti contraddicono, in parte, l'intuizione che vorremmo 
	concretizzare negli assiomi della teoria degli insiemi. Noi vorremmo \textbf{che un insieme fosse identificabile dalla sua struttura}. Mi spiego, per esempio $\emptyset$ è identificato dal fatto di non avere elementi,
	$\{\emptyset\}$ è identificato dal fatto di avere un solo elemento che non ha elementi etc. per tutti gli insiemi che conosciamo, ma cosa dire di $Y$? $Y$ ha un elemento $Y_1$, che ha un elemento $Y_2$,
	che ha \ldots e la stessa descrizione si potrebbe applicare anche a $Y_1$, e anche a $Y_2$ \ldots Sono tutti uguali? 
\end{enumerate}

Queste tre lacune saranno colmate dai tre assiomi che ancora ci mancano: rispettivamente l'assioma del rimpiazzamento, l'assioma della scelta e l'assioma di buona fondazione. La teoria risultante sarà,
inevitabilmente, incompleta - per esempio non decide il problema del continuo: l'esistenza di cardinalità intermedie fra $\aleph_0$ e $2^{\aleph_0}$ - ma è la fondazione meglio accettata della matematica.