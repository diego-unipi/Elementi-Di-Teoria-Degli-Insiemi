\section{I primi assiomi}
\subsection{Assiomi dell'insieme vuoto e di estensionalità}
\begin{axiom}
[Assioma dell'insieme vuoto]
\label{ax1}
Esiste un insieme vuoto.
\[ \exists x \; \forall y \; y \not\in x
		\]
\end{axiom}

\begin{note}
Questo assioma non sarebbe strettamente necessario, in quanto potremmo ottenere un insieme vuoto anche come sottoprodotto, per esempio, dell'assioma dell'infinito che vedremo in seguito.
Tuttavia è bello poter partire avendo per le mani almeno un insieme.
\end{note}

\begin{axiom}
[Assioma di estensionalità]
\label{ax2}
Un insieme è determinato dalla collezione dei suoi elementi. Due insiemi coincidono se e solo se hanno i medesimi elementi.
\[ \forall a \; \forall b \; a = b \leftrightarrow \forall x (x \in a \leftrightarrow x \in b)
	\]
\end{axiom}

\begin{exercise}
Dimostra che la freccia $a = b \rightarrow \forall x (x \in a \leftrightarrow x \in b)$, in realtà, segue dal fatto che se $a = b$ allora $a$ e $b$ sono indistinguibili\footnote{Nel senso che abbiamo descritto in precedenza, cioè sono nomi della stessa cosa.}.
\end{exercise}

\textbf{\underline{Convenzione}} Le variabili libere (= non quantificate), se non specificato altrimenti, si intendono quantificate universalmente all'inizio della formula. Per cui possiamo scrivere
l'assioma di estensionalità semplicemente nella forma:
\[ a = b \leftrightarrow \forall x (x \in a \leftrightarrow x \in b)
	\]

\begin{proposition}[Unicità dell'insieme vuoto]
C'è un unico insieme vuoto.
\[ \exists ! \, x \; \forall y \; y \not \in x
	\]
\end{proposition}

\begin{proof}
Consideriamo due insiemi vuoti $x_1$ e $x_2$, ossia supponiamo $\forall y \, y \not\in x_1$, e $\forall y \, y \not \in x_2$. Allora:
\[ \forall y (y \in x_1 \leftrightarrow y \in x_2)
	\]
[sono coimplicate logicamente] perché $y \in x_1$ e $y \in x_2$ sono entrambe necessariamente false (quindi la proposizione così com'è scritta è sempre vera). Per \hyperref[ax2]{estensionalità}, la proposizione sopra (sempre vera) è equivalente a $x_1 = x_2$ (che quindi a sua volta sarà sempre vera), e quindi abbiamo la tesi.
\end{proof}

\emph{Dimostrazione formale.} Questo livello di pedanteria non è necessario, ma, per una volta, proviamo a dimostrare in ogni dettaglio la formula $\exists ! x (\forall y (y \not \in x))$. Per definizione di $\exists !$, ciò equivale a:
\[ \exists x_1 ((\forall y \, y \not \in x_1) \land \forall x_2 ((\forall y \, y \not \in x_2) \rightarrow x_2 = x_1))
	\]
Per l'\hyperref[ax1]{assioma del vuoto}, $\exists x_1 \, \forall y \, y \not \in x_1$: fissiamo questo $x_1$. Resta da dimostrare che:
\[ (\forall y \, y \not \in x_1) \land \forall x_2(\forall y \, y \not \in x_2) \rightarrow x_2 = x_1
	\]
Per costruzione, $\forall y \, y \not\in x_1$, è vera (avendo fissato $x_1$), quindi resta:
\[ \forall x_2 (\forall y \, y \not \in x_2) \rightarrow x_2 = x_1
	\]
Ora prendiamo un $x_2$ qualunque, dobbiamo dimostrare:
\[ \forall y (y \not \in x_2) \rightarrow x_2 = x_1
	\]
Si danno due casi: o $\forall y (y \not \in x_2)$ è vera o è falsa. Nel secondo caso, l'implicazione è vera per via della tabella di verità. Nel primo abbiamo sia $\forall y \, y \not \in x_1$, [vera] per
costruzione, sia $\forall y \, y \not \in x_2$, [vera] per ipotesi. Quindi, preso un qualunque $y$, $y \in x_1$ e $y \in x_2$ sono entrambe false. La tabella di verità di $\leftrightarrow$ ci dice quindi che vale $y \in x_1 \leftrightarrow y \in x_2$, e, per 
l'arbitrarietà di $y$:
\[ \forall y (y \in x_1 \leftrightarrow y \in x_2)
	\]
Dall'\hyperref[ax2]{assioma di estensionalità}:
\[ \forall y (y \in x_1 \leftrightarrow y \in x_2) \rightarrow x_1 = x_2
	\]
Abbiamo quindi $x_1 = x_2$, da cui segue la verità dell'implicazione iniziale. $\hfill\square$


Chiaramente, ho voluto scrivere questa dimostrazione delirante per convincervi che NON È UNA BUONA IDEA.

\begin{notation}
L'unicità dell'insieme vuoto ci giustifica ad introdurre delle nuove abbreviazioni:
\[ x = \emptyset \Mydef \forall y \, y \not\in x \qquad \emptyset \in x \Mydef \exists z (z = \emptyset \land z \in x)
	\]
\end{notation}

\subsection{Assioma di separazione}
\begin{axiom}
[Assioma di separazione]
\label{ax3}
Se $A$ è un insieme, e $\psi(x)$ una formula insiemistica qualunque, allora $\{x \in A | \psi (x)\}$\footnote{Stiamo usando già questa notazione, ma la definiremo a breve.} è un insieme.
\[ \forall A \; \exists B \; \forall x \; x \in B \leftrightarrow (x \in A \land \psi (x))
	\]
\end{axiom}

\begin{note}
Tecnicamente l'assioma di separazione è uno \vocab{schema di assiomi}, ossia una regola che, per ogni possibile formula $\psi$, ci permette di scrivere un assioma.
\end{note}

\begin{proposition}
Fissati $A$ e $\psi(x)$, l'insieme $\{x \in A | \psi(x)\}$ è univocamente definito. Ossia:
\[ \forall A \; \exists \textcolor{red}{!} B \; \forall x \; x \in B \leftrightarrow (x \in A \land \psi(x))
	\]
\end{proposition}

\begin{proof}
Come per l'unicità dell'insieme vuoto, supponiamo di avere $B_1$ e $B_2$ tali che:
\[ \forall x \, x \in B_1 \leftrightarrow (x \in A \land \psi(x)) \qquad \forall x \, x \in B_2 \leftrightarrow (x \in A \land \psi(x))
	\]
Allora, $\forall x \, x \in B_1 \leftrightarrow (x \in A \land \psi(x)) \leftrightarrow x \in B_2$, quindi ciò coimplica, per \hyperref[ax2]{estensionalità}, che $B_1 = B_2$.
\end{proof}

\begin{exercise}[Transitività della coimplicazione]
Verificare che se $\psi \leftrightarrow \Phi$ e $\Phi \leftrightarrow \Theta$, allora $\psi \leftrightarrow \Theta$.
\end{exercise}

\begin{notation}
Vista l'unicità, possiamo introdurre una nuova abbreviazione:
\[ B = \{x \in A | \psi(x)\} \Mydef \forall x \, x \in B \leftrightarrow (x \in A \land \psi(x))
	\]
\end{notation}

Osserviamo che l'assioma di separazione è una forma indebolita del principio di collezione\footnote{Quel principio che definisce gli insiemi come tutte le cose che soddisfano una certa formula.}. Rimpiazzando il principio con questo assioma, il Paradosso di Russell diventa una proposizione.

\begin{proposition}[Insieme di tutti gli inisemi]
Non esiste l'insieme di tutti gli insiemi.
\[ \not\exists V \; \forall x \; x \in V
	\]
\end{proposition}

\begin{proof}
Supponiamo, per assurdo, che esista questo $V$. Allora, per \hyperref[ax3]{separazione} con la formula $\psi (x) \equiv x \not \in x$, esiste l'insieme:
\[ N = \{x \in V | x \not\in x\}
	\]
che, per definizione (via separazione), ha la proprietà:
\[ \forall x \, x \in N \leftrightarrow (x \in V \land x \not \in x)
	\]
Per ipotesi assurda, $x \in V$ è sempre vera (stiamo considerando l'insieme di tutti gli insiemi), quindi quanto scritto si riduce a:
\[ \forall x \, x \in N \leftrightarrow x \not\in x
	\]
prendendo ora come insieme $N$: $x = N$, abbiamo $N \in N \leftrightarrow N \not\in N$, assurdo.
\end{proof}

\subsection{Classi e classi proprie}
Sebbene, abbiamo detto che gli unici oggetti della teoria degli insiemi sono gli insiemi, usualmente ci si riferisce alla collezione di tutti gli insiemi 
che soddisfano una certa formula come ad una specie di insieme: una \vocab{classe}. Più precisamente, data una formula $\psi(x)$, se diciamo: ``sia $C$ la classe degli insiemi $x$ tali che $\psi(x)$''
intendiamo dire che useremo la scrittura $x \in C$ come una semplice abbreviazione per la formula $\psi(x)$.\footnote{Ovvero per tutti gli oggetti (solo gli insiemi in questo caso) che soddisfano una tale formula $\psi(x)$.} \\
Non avrebbe senso scrivere \textcolor{red}{$C \in$ qualcosa}, perché il simbolo $\in$ in $x \in C$ non ha senso (ha senso solo tra oggetti di tipo insieme), se non nel tutt'uno $\in C$. In altri termini, se scriviamo $x \in C$ in luogo di $\psi(x)$ è solo come ausilio dell'intuizione (per comodità insomma, senza intendere qualcosa di formale all'interno della teoria degli insiemi):
avremmo potuto decidere di scrivere $x$\ding{168}, o nient'altro che $\psi(x)$.

\begin{definition}[Classe universale]
La classe $V$ si dice \vocab{classe universale} ed è la classe di tutti gli insiemi.
\[ x \in V \Mydef x = x \footnote{Cioè la classe degli insiemi che soddisfano il predicato $\psi(x): x = x$ (ovvero tutti gli insiemi per quanto assunto all'inizio della teoria), $V = \{x | \psi(x)\} = \{x | x = x\}$ (dove naturalmente non sto usando separazione ma il principio di collezione perché stiamo definendo una classe).}
	\]
\end{definition}

Insomma, scrivere $x \in V$ non dice molto: è una formula sempre vera.

\begin{notation}[Uguaglianza tra classi]
Date due classi $C$ e $D$, che, ricordiamo, non significa altro che ``date due formule$\ldots$'', definiamo l'abbreviazione:
\[ C = D \Mydef \forall x ((x \in C) \leftrightarrow (x \in D)) \footnote{Non è altro che un abbreviazione per dire che le formule che definiscono le classi $C$ e $D$ sono soddisfatte dagli stessi insiemi $x$.}
	\]
\end{notation}

Ora, dato un qualunque insieme $A$, possiamo definire la classe $\hat{A}$ degli $x$ tali che $x \in A$ (cioè la classe degli $x$ che soddisfano $\psi(x) : x \in A$). Se $\hat{A} = \hat{B}$, per l'abbreviazione data non stiamo dicendo altro che:
\[ \forall x ((x \in A) \leftrightarrow (x \in B))
	\]
che equivale $A = B$ per \hyperref[ax2]{estensionalità}. Ha quindi senso, con un leggero abuso di notazione, omettere il cappelletto $\hat{}$ e ``identificare'' la classe $\hat{A}$ semplicemente con $A$. In questo senso,
abbiamo classi che sono insiemi - formalmente $C$ è un insieme se $C = \hat{A}$ per qualche insieme $A$ - e classi che non sono insiemi. Chiamiamo \vocab{classe propria} una classe che non è un insieme.\footnote{Essere un insieme per una classe significa quindi moralmente identificarvisi nel senso riportato sopra, se ciò non fosse possibile parliamo di classi proprie.}

\begin{example}
$V$ è una classe propria.
\end{example}

\textbf{\underline{L'intuizione}}, che sarà più chiara via via che procediamo nel corso, è che le classi proprie sono troppo grandi per essere insiemi.

\subsection{Assioma del paio e coppia di Kuratowski}
I primi tre assiomi ci dicono, a grandi linee, che, entro i limiti di quanto si può fare rinunciando al principio di collezione - che esiste $\{x | \, \text{una qualunque proprietà}\}$ -, gli insiemi sono delle specie di collezioni.
Sono determinati dai loro elementi, e li si può dividere in collezioni più piccole in maniera arbitraria. \\ Ci troviamo, però, adesso, nella necessità di procurarci qualche insieme con cui lavorare. I prossimi assiomi serviranno per giustificare le costruzioni con cui,
usualmente, si definiscono nuovi insiemi. Per esempio, abbiamo bisogno di costruire certi insiemi di base, tipo l'insieme dei numeri interi o insiemi finiti i cui elementi sono elencati esplicitamente, fare prodotti di insiemi esistenti, 
considerare le funzioni fra insiemi esistenti, etc.

\begin{axiom}
[Assioma del paio]
\label{ax4}
Dati $a$ e $b$ esiste l'insieme $\{a,b\}$.
\[ \forall a \; \forall b \; \exists P \; \forall x \; x \in P \leftrightarrow (x = a \lor x = b)
	\]
\end{axiom}

\begin{proposition}
[Unicità del paio]
Fissati $a$ e $b$, l'insieme $\{a,b\}$ è univocamente determinato.
\[\forall a \; \forall b \; \exists\textcolor{red}{!} P \; \forall x \; x \in P \leftrightarrow (x = a \lor x = b)
	\]
\end{proposition}

\begin{exercise}
	Dimostra la proposizione precedente.
\end{exercise}

\begin{soln}
	Supponiamo che esistano $P_1$ e $P_2$ tali che:
	\[ \forall x (x \in P_1 \leftrightarrow (x = a \lor x = b)) \qquad \text e \qquad \forall x (x \in P_2 \leftrightarrow (x = a \lor x = b))
		\]
	da ciò segue che:
	\[ \forall x (x \in P_1 \leftrightarrow x \in P_2)
		\]
	dunque per \hyperref[ax2]{estensionalità} l'espressione sopra equivale a $P_1 = P_2$.
\end{soln}

\begin{proposition}[Esistenza dei singoletti]
	Dato $a$, esiste ed è unico $\{a\}$.
	\[ \forall a \; \exists ! S \; \forall x \; x \in S \leftrightarrow x = a
		\]
\end{proposition}

\begin{proof}
	Ponendo $b = a$ nella proposizione precedente, si ha che:
	\[ \forall a \; \exists ! S \; \forall x \; x \in S \leftrightarrow (x = a \lor x= a)
		\]
	ora $x = a \lor x = a$ equivale a $x = a$\footnote{Stiamo dicendo che in generale $\{a,a\} = \{a\}$ poiché $a \lor a = a$ (in base alle regole dei connettivi logici).}.
\end{proof}

\begin{notation}[Paio (o coppia) e singoletto]
	Possiamo ora introdurre delle abbreviazioni per il paio (o coppia) ed i singoletti:
	\[ P = \{a,b\} \Mydef \forall x \, x \in P \leftrightarrow (x = a \lor x = b)
		\]\[ S = \{a\} \Mydef \forall x \, x \in S \leftrightarrow x = a
			\]
\end{notation}

\begin{remark}
	Osserviamo che $\{a,b\} = \{b,a\}$.
\end{remark}

\begin{proof}
	Segue dal fatto che $\lor$ è commutativo:
	\[ x \in \{a,b\} \leftrightarrow (x = a \lor x = b) \leftrightarrow (x = b \lor x = a) \leftrightarrow x \in \{b,a\}
		\]
	quindi per \hyperref[ax2]{estensionalità} $\{a,b\} = \{b,a\}$.
\end{proof}

Il paio $\{a,b\}$ è, quindi, una coppia non ordinata. È possibile codificare le coppie ordinate con il seguente trucco.

\begin{definition}
	[Coppia di \href{https://it.wikipedia.org/wiki/Kazimierz_Kuratowski}{\textcolor{purple}{Kuratowski}}]
	Definiamo la \vocab{coppia di Kuratowski}:
	\[(a,b) \Mydef \{{a},\{a,b\}\}
		\]
\end{definition}

\begin{proposition}[Proprietà di coppia ordinata]
	La coppia di Kuratowski $(a,b)$ rappresenta la coppia ordinata di $a$ e $b$, ossia vale che:
	\[ (a,b) = (a^{\prime},b^{\prime}) \leftrightarrow (a = a^{\prime} \land b = b^{\prime})
		\]
\end{proposition}

\begin{proof}
	Detto $c = (a,b)$, vogliamo determinare univocamente $a$ e $b$. Osserviamo che $a$ è determinata da:
	\[ x = a \leftrightarrow \forall y \in c (x \in y) \, \footnote{Sostanzialmente stiamo dicendo che $a$ è identificato univocamente come l'elemento che appartiene ad entrambi gli elementi di $(a,b)$.}
		\]
	la freccia $\rightarrow$ segue da come è definita la coppia $(a,b)$, mentre $\leftarrow$ segue dal fatto che, sempre per definizione di coppia di Kuratowski, $\{a\} \in c = (a,b)$, per cui:
	\[ \forall y \in c (x \in y) \overset{\text{ipotesi}}{\implies} x \in \{a\} \overset{\text{singoletto}}{\implies} x = a
		\]
	Determiniamo ora univocamente $b$. Studiamo prima il caso in cui $\exists ! x (x \in c)$ - cioè il caso in cui abbiamo $(x,x) = \{\{x\}\}$ -:
	\[	\exists ! x (x \in c) \iff \{a\} = \{a,b\} \iff b = a
		\]
	In questo caso $b$ è determinato. Se non fosse così allora $\{a,b\}$ avrebbe due elementi distinti e $b$ sarebbe univocamente determinato da:
	\[ x = b \leftrightarrow (x \in \{a,b\} \land x \ne a)
		\]
\end{proof}

\begin{definition}[$n$-upla ordinata]
	Possiamo estendere la definizione di coppia ordinata con il seguente trucco:
	\begin{align*}
	   (a,b,c) &\Mydef ((a,b),c) \\
	   (a,b,c,d) &\Mydef (((a,b),c),d) \\
	   (a_1,a_2,\ldots,a_n) &\Mydef ((a_1,a_2,\ldots,a_{n-1}),a_n)
	\end{align*}
\end{definition}

\begin{note}
	Quest'ultima definizione è, in realtà, uno schema di definizioni: una per ogni $n$. Per ora, \textcolor{red}{NON} siamo in grado di scrivere, per esempio,
	una formula insiemistica che dica ``Esiste un $n$ ed una $n$-upla $(a_1,\ldots,a_n)$ tale che…''. Però, per ogni $n$ dato, chessò 92, possiamo scrivere esplicitamente una formula che dice $x = (a_1,a_2,a_3,\ldots,a_{92})$.
\end{note}

\begin{proposition}[Proprietà di $n$-upla ordinata]
	Si ha che:
	\[ (a,b,c) = (a',b',c') \leftrightarrow a = a' \land b = b' \land c = c'
		\]\[ (a_1,\ldots,a_n) = (a_1',\ldots,a_n') \leftrightarrow a_1 = a_1' \land \ldots \land a_n = a_n'
			\]
\end{proposition}

\begin{exercise}
	Dimostra la prima e convinciti che, dato un qualunque $n$ esplicito, potresti dimostrare la seconda.
\end{exercise}

\subsection{Assioma dell'unione e operazioni booleane}

\begin{axiom}[Assioma dell'unione]
	\label{ax5}
	Dato un insieme $A$ esiste un insieme $B$ i cui elementi sono gli elementi degli elementi di $A$. Ovvero, dato un insieme $A$ esiste l'unione degli elementi di $A$.
	\[ \forall A \; \exists B \; \forall x \; x \in B \leftrightarrow \exists y \in A \; x \in y\footnote{Cioè $x$ è un elemento di $B$ se e solo se è un elemento di un elemento di $A$.}
		\]
\end{axiom}

\begin{proposition}
	[Unicità dell'unione]
	Vale l'unicità dell'unione:
	\[ \forall A \; \exists \textcolor{red}{!} B \; \forall x \; x \in B \leftrightarrow \exists y \in A \; x \in y
		\]
\end{proposition}

\begin{proof}
	Supponiamo di avere $B_1$ e $B_2$ tali che:
	\[ \forall x \, x \in B_1 \leftrightarrow \exists y \in A \, x \in y
		\]\[ \forall x \, x \in B_2 \leftrightarrow \exists y \in A \, x \in y
			\]
	quindi $\forall x (x \in B_1 \leftrightarrow x \in B_2)$, e per \hyperref[ax2]{estensionalità} $B_1 = B_2$.
\end{proof}

\begin{notation}[Unione di un insieme]
	Possiamo introdurre l'abbreviazione:
	\[ B = \bigcup A\footnote{\,``Unione di $A$''.} \Mydef \forall x \,( x \in B \leftrightarrow \exists y (x \in y))
		\]
\end{notation}

\begin{exercise}
	Dimostra che l'assioma dell'unione segue che:
	\[ \forall A \; \exists B \; (\forall y \in A \; \forall x \in y \; x \in B)\footnote{Cioè per ogni insieme esiste l'insieme di tutti gli elementi degli elementi di $A$.}
		\]
\end{exercise}

Combinando l'assioma dell'unione e del paio possiamo definire $a \cup b$.

\begin{definition}[Unione di insiemi]
	Poniamo:
	\[ a \cup b \Mydef \bigcup\{a,b\}
		\]
\end{definition}

\begin{proposition}[Caratterizzazione unione di insiemi]
	Dati $a,b$ e $a \cup b$ vale che:
	\[ x \in a \cup b \leftrightarrow (x \in a \lor x \in b)
		\]
\end{proposition}

\begin{proof}
	Dire che $x$ è un elemento di $a \cup b$ equivale a dire che $x$ è un elemento di un elemento di $\{a,b\}$, ossia
	che $x$ è un elemento di uno tra $a$ e $b$ ($x \in a \lor x \in b$).
\end{proof}

Ora definiamo le intersezioni: \emph{riesci a vedere perché, a differenza delle unioni, non servirà un nuovo assioma?}

\begin{definition}[Intersezione di un insieme]
	Sia $C$ una \textcolor{red}{classe}\footnote{Quindi, in particolare, $C$ può essere un insieme (in questo caso la definizione è comunque lecita in generale con le classi, i cui elementi sono  appunto insiemi).} non vuota.
	L'\textcolor{red}{insieme} $B$ è l'\vocab{intersezione} di $C$ se:
	\[ B = \bigcap C \Mydef \forall x (x \in B \leftrightarrow \forall y \in C (x \in y))
		\]
	cioè $x$ sta in $b$ se è elemento di ogni elemento di $C$.
\end{definition}

\begin{proposition}[Esistenza e unicità dell'intersezione]
	Data una classe non vuota $C$, l'intersezione $\bigcap C$ esiste [cioè è un insieme] ed è unica. In particolare, nel caso dell'intersezione di un insieme vale:
	\[ \forall A (A \ne \emptyset \rightarrow \exists ! B \; \forall x(x \in B \leftrightarrow \forall y \in A (x \in y)))
		\]
\end{proposition}

\begin{note}
	L'ipotesi $C \ne \emptyset$ è necessaria perché altrimenti si avrebbe che $\bigcap \emptyset$ è la classe universale $V$ ($x \in \bigcap \emptyset \leftrightarrow \forall y \in \emptyset(x \in y)$ (dove il RHS è sempre falso per costruzione, quindi gli $x$ che soddisfano l'enunciato sono tutti)), che non è un insieme.
\end{note}

\begin{proof}
	L'unicità segue per \hyperref[ax2]{estensionalità} al solito modo. Veniamo all'esistenza. Dal momento che $C$ non è vuota esiste $z \in C$,
	dunque per separazione possiamo costruire l'insieme:
	\[ B := \{x \in z | \forall y \in C (x \in y)\}
		\]
	e verificare che tale insieme è proprio l'intersezione che stiamo cercando di costruire. Infatti, $x \in B \implies \forall y \in C (x \in y)$, d'altro canto, $\forall y \in C(x \in y)$ implica,
	in particolare, $y \in z$, per cui $y \in B$. Abbiamo così verificato che $x \in B \leftrightarrow \forall y \in C (x \in y)$, ossia $B = \bigcap C$.
\end{proof}

\begin{notation}[Intersezione e differenza di insiemi]
	Poniamo:
	\[ a \cap b \Mydef \bigcap\{a,b\} \qquad \text e \qquad a\setminus b \Mydef \{x \in a | x \not\in b\}
		\]
\end{notation}

\begin{proposition}[Caratterizzazione intersezione e differenza di insiemi]
	Vale che:
	\[ x \in a \cap b \leftrightarrow (x \in a \land x \in b)
		\]\[ x \in a \setminus b \leftrightarrow (x \in a \land x \not\in b)
			\]
\end{proposition}

\begin{exercise}
	Dimostrare la proposizione precedente (la seconda è semplicemente la definizione).
\end{exercise}

\begin{proposition}[Proprietà di unione, intersezione e differenza di insiemi]
	Alcune proprietà delle operazioni $\cup$, $\cap$, $\setminus$:
	\[ \begin{split}
		\text{\textcolor{red}{commutatività:}} \qquad & a \cup b = b \cup a \qquad \text e \qquad a \cap b = b \cap a \\
		\text{\textcolor{red}{associatività:}} \qquad & a \cup (b \cup c) = (a \cup b) \cup c \Mydef a \cup b \cup c \\
	                                        	      & a \cap (b \cap c) = (a \cap b) \cap c \Mydef a \cap b \cap c \\
		\text{\textcolor{red}{distributività:}} \qquad & a \cup (b \cap c) = (a \cup b) \cap (a \cup c) \\
													   & a \cap (b \cup c) = (a \cap b) \cup (a \cap c) \\
	    \text{\textcolor{red}{leggi di \href{https://it.wikipedia.org/wiki/Augustus_De_Morgan}{\textcolor{red}{De Morgan}}:}} \qquad & a \setminus (b \cup c) = (a \setminus b) \cap (a \setminus c) \\
																													& a \setminus (b \cap c) = (a \setminus b) \cup (a \setminus c)
	   \end{split}
	\]
\end{proposition}

\begin{proof}
	Tutte queste proprietà su deducono immediatamente dalle corrispondenti proprietà dei connettivi logici, le quali, a loro volta, si vedono con le tabelle di verità. Per esempio, dimostriamo 
	la prima delle leggi di De Morgan (facendo uso della corrispondente legge per i connettivi logici):
	\[ \begin{split}
		x \in a \setminus (b \cup c) \iff & x \in a \land x \not\in (b \cup c)\\
		\iff & x \in a \land \neg(x \in b \lor x \in c)\\
		\overset{\text{De Morgan}}{\iff} & x \in a \land x \not\in b \land x \not\in c\\
		\iff & x \in a \land x \not\in b \land \underbrace{x \in a}_{\text {non cambia nulla}} \land x \not\in c\\
		\iff & x \in (a \setminus b) \land x \in (a \setminus c)\\
		\iff & x \in (a \setminus b) \cap (a \setminus c)
	\end{split}
		\]
\end{proof}

Ora possiamo costruire insiemi finiti elencandone gli elementi, come si fa di solito, con la notazione $\{\ldots\}$\footnote{Paradossalmente prima di aggiungere l'assioma dell'unione
alla teoria potevamo costruire $n$-uple ordinate di lunghezza arbitraria, ma non un insieme con più di due elementi.}.
\pagebreak
\begin{notation}[Insiemi di $n$ elementi]
	Possiamo ora introdurre un'abbreviazione per indicare insiemi con più di due elementi (costruiti usando l'\hyperref[ax5]{assioma dell'unione}):
	\[ \begin{split}
	   \{a,b,c\} &\Mydef \{a\} \cup \{b\} \cup \{c\} \\
	   \{a,b,c,d\} &\Mydef \{a\} \cup \{b\} \cup \{c\} \cup \{d\} \\
	   \{a_1,\ldots,a_n\} &\Mydef \{a_1\} \cup \ldots \cup \{a_n\}
	\end{split}
	\]
\end{notation}

\begin{proposition}[Caratterizzazione di insieme con $n$ elementi]
	Vale che:
	\[  \begin{split}
		x \in \{a,b,c\} & \leftrightarrow (x = a \lor x = b \lor x = c) \\
	    x \in \{a_1,\ldots,a_n\} & \leftrightarrow (x = a_1 \lor \ldots \lor x = a_n) 
		\end{split}
			\]
\end{proposition}

\begin{exercise}
	Dimostrare la proposizione precedente.
\end{exercise}

\subsection{Assioma delle parti e prodotto cartesiano}
Abbiamo definito le coppie $(x,y)$, però, per esempio, ancora nulla ci assicura che dati $A$ e $B$ esista:
\[ A \times B = \{(x,y) | x \in A \land y \in B\}
	\]
Le funzioni $A \rightarrow B$ saranno poi sottoinsiemi di $A \times B$, e vorremo parlare dell'insieme ${}^{A}B$
delle funzioni $A \rightarrow B$. Per tutto questo ci manca un solo ingrediente: l'insieme delle parti.

\begin{axiom}
	[Assioma delle parti]
	\label{ax6}
	Dato un insieme $A$ esiste l'insieme $\ps(A)$ i cui elementi sono i sottoinsiemi di $A$.
	\[ \forall A \; \exists B \; \forall x \; x \in B \leftrightarrow x \subseteq A 
		\]
\end{axiom}

\begin{proposition}
	[Unicità delle parti]
	Vale che:
	\[\forall A \; \exists ! B \; \forall x \; x \in B \leftrightarrow x \subseteq A 
		\]
\end{proposition}

\begin{proof}
	Segue come sempre per \hyperref[ax2]{estensionalità}, in quanto, se avessimo $B_1$, $B_2$, allora:
	\[ \forall x (x \in B_1 \leftrightarrow x \subseteq A) \qquad \text e \qquad \forall x(x \in B_2 \leftrightarrow x \subseteq A)
		\]
	quindi $\forall x((x \in B_1) \leftrightarrow (x \subseteq A) \leftrightarrow (x \in B_2)) \leftrightarrow \forall x (x \in B_1 \leftrightarrow x \in B_2) \leftrightarrow B_1 = B_2$.
\end{proof}

\begin{notation}[Insieme delle parti - o insieme potenza]
	Data l'unicità possiamo porre:
	\[ B = \ps(A) \Mydef \forall x \; x \in B \leftrightarrow x \subseteq A
		\]
\end{notation}

\begin{proposition}[Esistenza ed unicità del prodotto cartesiano]
	Dati $A$ e $B$  esiste ed è unico insieme $A \times B$ tale che:
	\[ \forall z (z \in A \times B) \leftrightarrow \exists x \in A \, \exists y \in B \, z = (x,y)\footnote{Ossia, informalmente, $ z \in A \times B$ se e solo se si può scrivere come coppia ordinata di un elemento di $A$ ed uno di $B$.}
		\]
\end{proposition}

\begin{proof}
	L'unicità è conseguenza immediata della definizione e dell'\hyperref[ax2]{assioma di estensionalità} (stessa dimostrazione di sempre). Per l'esistenza, definiamo per \hyperref[ax3]{separazione}:
	\[ A \times B \Mydef \{z \in \ps(\ps(A \cup B)) | \exists x \in A \ \exists y \in B \ z = (x,y)\}
		\]
	così come scritto, siamo sicuri che è un insieme che contiene coppie ordinate di elementi di $A$ e $B$, tuttavia, affinché tale insieme che abbiamo costruito nella teoria, rispetti la definizione data di prodotto cartesiano, dobbiamo anche dimostrare anche che ogni coppia $(x,y)$ con $x \in A$ e $y \in B$ appartiene a questo insieme.
	Per fare ciò bisogna dimostrare che tutte queste coppie appartengono a $\ps(\ps(A \cup B))$:\footnote{Poniamo $a,b,\ldots \in z \Mydef a \in z \land b \in z \land \ldots$ e $a,b,\ldots \subseteq z \Mydef a \subseteq z \land b \subseteq z \land \ldots$}\,\footnote{Tutte le implicazioni si basano sul fatto che se un oggetto è sottoinsieme di un qualche insieme allora è un elemento del corrispondente insieme delle parti per definizione.}
	\[\begin{split}
		a \in A \land b \in B &\implies \{a\},\{a,b\} \subseteq A \cup B \\
		& \implies \{a\},\{a,b\} \in \ps(A \cup B) \\
		& \overset{\text{\hyperref[ax4]{paio}}}{\implies} (a,b) = \{\{a\},\{a,b\}\} \subseteq \ps(A \cup B)\\
		& \implies (a,b) \in \ps(\ps(A \cup B))
	\end{split}
		\]
	pertanto tutte le coppie ordinate di elementi di $A$ e $B$ appartengono a $A \times B$, che quindi rispetta proprio la definizione di prodotto cartesiano voluta.
\end{proof}

\begin{note}
	Avremmo potuto costruire $A \times B$ usando, anziché l'assioma delle parti, l'assioma del rimpiazzamento, che vedremo più avanti.
\end{note}

\subsection{Relazioni di equivalenza e di ordine, funzioni}
Ora rivedremo alcuni concetti ben noti dai primi corsi del primo anno (\emph{o dalla scuola superiore?}). Lo facciamo molto rapidamente, essenzialmente per completezza, e per fissare le notazioni.

\begin{definition}[Relazione binaria]
	Si dice \vocab{relazione binaria} fra $A$ e $B$ un sottoinsieme di $A \times B$.
\end{definition}

\begin{notation}[Relazione binaria]
	Data una relazione $\rel \subseteq A \times B$, definiamo l'abbreviazione:
	\[ a \rel b \Mydef (a,b) \in \rel
		\]
\end{notation}

\begin{example}
	Per esempio scriviamo $a < b$ per indicare che $(a,b) \in\, <$.
\end{example}

Considerando il caso di $A \times A$ possiamo definire le seguenti relazioni.

\begin{definition}
	Una relazione $\sim \,\subseteq A \times A$ è una \vocab{relazione di equivalenza} se è:
	\begin{enumerate}[(i)]
		\item \textbf{\underline{riflessiva}}: $\forall x \in A \; x \sim x$.
		\item \textbf{\underline{simmetrica}}: $\forall x,y \in A\footnote{$\forall x_1,\ldots,x_n \Mydef \forall x_1 \ldots \forall x_n$, e lo stesso con $\exists$ e con i quantificatori limitati.} x \sim y \leftrightarrow y \sim x$.
		\item \textbf{\underline{transitiva}}: $\forall x,y,z \in A \; (x \sim y \land y \sim z) \rightarrow x \sim z$.
	\end{enumerate}
\end{definition}

\begin{definition}
	$\leq \, \in A \times A$ è una \vocab{relazione di ordine (largo)} se è:
	\begin{enumerate}[(i)]
		\item \textbf{\underline{riflessiva}}: $\forall x \in A \; x \leq x$.
		\item \textbf{\underline{antisimmetrica}}: $\forall x,y \in A \; (x \leq y \land y \leq x) \rightarrow x = y$.
		\item \textbf{\underline{transitiva}}: $\forall x,y,z \in A \; (x \leq y \land y \leq z) \rightarrow x \leq z$.
	\end{enumerate}
\end{definition}

\begin{definition}
	$< \, \in A \times A$ è una \vocab{relazione di ordine stretto} se è:
	\begin{enumerate}[(i)]
		\item \textbf{\underline{irriflessiva}}: $\forall x \in A \; \neg(x < x)$.
		\item \textbf{\underline{transitiva}}: $\forall x,y,z \in A \; (x < y \land y < z) \rightarrow x < z$.
	\end{enumerate}
\end{definition}

\begin{exercise}
	Dimostra che una relazione di ordine stretto $<$ su $A$ è automaticamente asimmetrica:
	\[ \forall x,y \in A \; x < y \rightarrow \neg (y < x)
		\]
\end{exercise}

\begin{soln}
Se valesse che $\forall x,y \in A \; x < y \rightarrow y < x$, allora sarebbero contemporaneamente vere $x < y$ e $y < x$, da cui, per transitività si avrebbe $x < x$ che è falso.
\end{soln}

\begin{proposition}[Corrispondenza tra ordini stretti e larghi]
	Data una relazione di ordine stretto $<$ su $A$, la relazione:
	\[ \leq\, = \{(x,y) \in A \times A | x < y \lor x = y\}\footnote{Formalmente: $\{z \in A \times A | \exists x,y \in A \; z = (x,y) \land \ldots\}$.}
		\]
	è una relazione di ordine largo. Viceversa, se $\leq$ è una relazione di ordine largo, la seguente relazione è dei ordine stretto:
	\[ <\, = \{(x,y) \in A \times A | x \leq y \land x \ne y\}\footnote{Come la nota sopra.}
		\]
	Inoltre, in questo modo, le relazioni di ordine stretto e di ordine largo sono poste in corrispondenza una - a - uno.
\end{proposition}

\begin{proof}
	Definiamo la \vocab{diagonale di una relazione} di $A \times A$ come:
	\[ \Delta_A \Mydef \{(x,y) \in A \times A | x = y\}
		\]
	Allora è facile verificare che, se $<$ è una relazione di ordine stretto, allora $< \cap \,\Delta_A = \emptyset$ e $< \cup \,\Delta_A$ è una relazione di ordine largo corrispondente.
	Viceversa, se $\leq$ è una relazione di ordine largo, allora $\Delta_A \subseteq \, \leq$ e $\leq \setminus \Delta_A$ è la relazione di ordine stretto corrispondente.
\end{proof}

\begin{notation}[Relazioni d'ordine strette e larghe]
	Fissata una relazione di ordine largo $\textcolor{red}{\leq}$ su $A$, ci sentiremo liberi di usare la corrispondente relazione di ordine stretto $\textcolor{red}{<}$ fintanto che la scelta del simbolo sia indizio sufficiente dell'operazione.
	Inoltre scriveremo $x > y$ per $y < x$ e $x \geq y$ per $y \leq x$.
\end{notation}

\begin{definition}[Relazione di ordine totale]
	Una \vocab{relazione di ordine totale} su $A$ è una relazione di ordine $\leq$ tale che:
	\[ \forall x,y \in A \, (x \leq y) \lor (x = y) \lor (y \leq x)
		\]
\end{definition}

\begin{exercise}
	Formula la definizione precedente per ordini stretti.
\end{exercise}

\begin{soln}
Diciamo che $<$ è un ordinamento totale (stretto) su $A$ se:
\[ \forall x \in A \, \forall y \in A (x \ne y \land ((x < y) \lor (x > y))) \lor (x = y)
	\]
o anche semplicemente:
\[ \forall x \in A \, \forall y \in A (x = y) \lor (x < y) \lor (x > y)
	\]
E per quanto detto possiamo anche pensare che:
\[ \text{$\leq$ ordine totale} \iff \text{$< \cup \Delta_A$ ordine totale}
	\]
(infatti nella prima definizione non è strettamente necessario che compaia l'uguaglianza, la si può ottenere quanto entrambe le disuguaglianze sono vere per antisimmetria, mentre per ordini stretti è necessario aggiungere la diagonale nella definizione di totalità).
\end{soln}

\begin{definition}[Restrizione di una relazione]
	Data una relazione $\rel \subseteq A \times B$, e dati $A' \subseteq A$, $B' \subseteq B$, possiamo definire la \vocab{restrizione} di $\rel$ a $A'\times B'$:
	\[ \rel_{|A' \times B'} \Mydef \rel \cap (A' \times B')
		\]
	``restrizione di $\mathcal R$ a $A' \times B'$\,''.
\end{definition}

\begin{exercise}
	Data $\rel$ relazione di equivalenza/ordine su $A$ e $A' \subseteq A$, dimostra che $\rel_{|A' \times A'}$ è una relazione di equivalenza/ordine su $A'$.
\end{exercise}

\begin{soln}
Vediamolo per le relazioni di equivalenza. È facile osservare che $\forall a' \in A'$, vale che $(a',a') \in \mathcal{R}_{|A' \times A'}$ (sta in $A' \times A'$ per definizione di prodotto cartesiano e sta in $\mathcal{R}$ essendo una relazione di equivalenza per ipotesi (vale il per ogni)),
analogamente valgono simmetria e riflessività. 
\end{soln}

\begin{definition}[Dominio e immagine di una relazione]
	Data una relazione $\rel \subseteq A \times B$, definiamo:
	\[  \begin{split}
		\Dom(\rel) &\Mydef \{x \in A | \exists y \in B \; x\rel y\} \qquad \text{\vocab{dominio} di $\rel$} \\
	    \Imm(\rel) &\Mydef \{y \in B | \exists x \in A \; x \rel y \} \qquad \text{\vocab{immagine} di $\rel$}
	\end{split}
			\]
	(notare che $\Dom(\rel)$ e $\Imm(\rel)$ non coincidono necessariamente con $A$ e $B$).
\end{definition}

\begin{definition}[Funzione]
	Chiamiamo \vocab{funzione} $f:A \rightarrow B$ una relazione $f \subseteq A \times B$ tale che:
	\[ \forall x \in A \; \exists ! \, y \in B \; (x,y) \in f
		\]
		(Intuitivamente $f$ è l'insieme delle coppie $(x,f(x))$ per $x \in A$).
\end{definition}

\begin{notation}[Immagine e immagine di un sottoinsieme]
	Data una funzione $f$ possiamo indicare la coppia $(x,y) \in f$ con la seguente abbreviazione:
	\[ y = f(x) \Mydef (x,y) \in f
		\]
	Dato $S \subseteq \Dom(f)$, indichiamo l'immagine di un sottoinsieme (ovvero l'insieme delle immagini del sottoinsieme) come:
	\[ f[S] \Mydef \{y \in \Imm(f)| \exists x \in S \; \underbrace{y = f(x)}_{= (x,y) \in f}\} = \underbrace{\{f(x) | x \in S\}}_{\text{informalmente}}
		\]
\end{notation}

\begin{definition}[Iniettività, suriettività e bigettività]
	Una funzione $f: A \rightarrow B$ è:
	\[ \begin{split}
		\text{\vocab{iniettiva} se:}\; & \forall y \in \Imm(f)\; \exists! \, x \in \Dom(f) \; f(x) = y\\
		\text{\vocab{suriettiva} se:}\; & B = \Imm(f)\; \text{ossia $\forall y \in B \; \exists x \in A \; f(x) = y$.}\\
		\text{\vocab{bigettiva} se:}\; &\text{è sia iniettiva sia surgettiva.}
	\end{split}
		\]
\end{definition}

\begin{definition}[Funzione inversa]
	Data $f$ iniettiva:
	\[ f^{-1} \Mydef \{(y,x) \in B \times A | f(x) = y\} \subseteq B \times A
		\]
\end{definition}

\begin{remark}[Funzione inversa e controimmagine]
	Se $f$ iniettiva, $f^{-1} : \Imm(f) \rightarrow \Dom(f)$ è una funzione\footnote{Altrimenti è la semplice controimmagine di un sottoinsieme dell'immagine (che non è una funzione).} a
	sua volta iniettiva (basta pensare alla definizione di $f^{-1}$ iniettiva e usare che per l'iniettività di $f$ c'è un'unica $x \in \Dom(f)$ tale che $y = f(x)$).
	In particolare se $f : A \rightarrow B$ è bigettiva, allora $f^{-1}$ è bigettiva.
\end{remark}

\begin{definition}[Restrizione di una funzione]
	Data $f: A \rightarrow B$ e $A' \subseteq A$ definiamo:
	\[ f_{|A'} \Mydef \{(x,y) \in A' \times B | f(x) = y\}
		\]
	 ``$f$ \vocab{ristretta} ad $A'$\,'' è una funzione: $A' \rightarrow B$.
\end{definition}

\begin{definition}[Composizione di funzioni]
	Date $g : A \rightarrow B$ e $f : B \rightarrow C$:
	\[ f \circ g \Mydef \{(x,z) \in A \times C | z = f(g(x))\}\footnote{O più formalmente $\exists y(y = g(x) \land z = f(y))$.}
		\]
	``$f$ \vocab{composta} con $g$'' è una funzione: $A \rightarrow C$.
\end{definition}

\begin{notation}[Funzione identità]
	Indichiamo con $\id_A$ la \vocab{funzione identità} su $A$:
	\[ \id_A \Mydef \{(x,y) \in A \times A | x = y\} = \Delta_A
		\]
\end{notation}

\begin{remark}[Caratterizzazione funzione inversa]
	Data $f : A \rightarrow B$ bigettiva e $g : B \rightarrow A$ è equivalente scrivere:
	\[ g = f^{-1} \qquad g \circ f = \id_A \qquad f \circ g = \id_B
		\]
\end{remark}

\begin{exercise}[Composizione di funzioni iniettive/surgettive/bigettive]
	Data $f : A \rightarrow B$ e $g: B \rightarrow C$, sotto quali condizioni $g \circ f$ è iniettiva, suriettiva, bigettiva?
\end{exercise}

\begin{soln}
	Indaghiamo il problema partendo prima dalle singole funzioni con delle proprietà e componendole. Se $f$ e $g$ sono iniettive, allora $g \circ f$ è iniettiva, infatti:
	\[ g(f(x)) = g(f(y)) \overset{\text{$g$ iniett.}}{\iff} f(x) = f(y) \overset{\text{$f$ iniett.}}{\iff} x = y \qquad \forall x,y \in A
		\]
		che è equivalente alla definizione di $g \circ f : A \rightarrow C$ iniettiva. Se $f$ e $g$ sono surgettive, allora $g \circ f$ è surgettiva:
	\begin{align*}
		\text{$g$ surgettiva} &\iff \forall z \in C \; \exists y \in B \; g(y) = z \\
		\text{$f$ surgettiva} &\iff \forall y \in B \; \exists x \in A \; f(x) = y
	\end{align*}
	che messe assieme ci danno che $g(f(x)) = z$, cioè per ogni $z \in C$ esiste $x \in A$ tale che $(g \circ f)(x) = z$, che è equivalente alla definizione di $g \circ f$ surgettiva.
	Naturalmente, mettendo assieme i risultati precedenti, otteniamo che $f$ e $g$ bigettive implica $g \circ f$ bigettiva. Viceversa, osserviamo che se $g \circ f$ è iniettiva, allora $f$ è iniettiva,
	infatti, se per assurdo $f(x) = f(y)$, con $x \ne y$, allora, applicando $g$, si ha $g(f(x)) = g(f(y))$ (perché immagini di cose uguali), ma per iniettività di $g \circ f$, ciò equivale a $x = y$,
	che è assurdo, pertanto $x = y$\footnote{Abbiamo dimostrato per assurdo che $f(x) = f(y) \implies x = y$ (sotto l'ipotesi che $g \circ f$ iniettiva), il viceversa è banale e con questo si ha l'equivalenza con la definizione di $f$ iniettiva}.
	Se $g \circ f$ è surgettiva, allora $g$ è surgettiva, infatti, per ipotesi, $\forall z \in C \; \exists x \in A \; g(f(x)) = z$, e, dato che $f(x) \in B$, abbiamo trovato che per ogni $z \in C$ esiste $y = f(x) \in B$ tale che $g(y) = z$, ovvero $g$ surgettiva.\\
	Infine, verrebbe da chiedersi, se date $f$ iniettiva e $g$ surgettiva, $g \circ f$ sia necessariamente bigettiva (così da avere magari un'equivalenza tra la bigettività della composizione e le proprietà delle funzioni in partenza), sfortunatamente ciò è falso: presa
	$f : \{0,1\} \hookrightarrow \{0,1,2,3\}$ e $g : \{0,1,2,3\} \twoheadrightarrow \{0,1,2\}$, con:
	\begin{align*}
		& g(0) = 0 \qquad f(0) = 0 \\
		& g(1) = 0 \qquad f(1) = 1 \\
		& g(2) = 2 \\
		& g(3) = 3
	\end{align*}
	abbiamo $f$ iniettiva, $g$ surgettiva, ma $g \circ f$ non è né iniettiva ($g(f(0)) = g(f(1))$) né surgettiva ($\Imm(g \circ f) = \{0\}$).
\end{soln}

\begin{exercise}[Insieme quoziente e proiezione]
	\label{3.73}
	Data una relazione di equivalenza $\sim$ su $A$, dimostra che esiste un insieme $\faktor{A}{\sim}$ ed una funzione surgettiva $i_\sim$ da $A$ a $\faktor{A}{\sim}$
	tale che:
	\[ \forall x,y \in A \; x \sim y \leftrightarrow i_\sim(x) = i_\sim(y)
		\]
\end{exercise}

\begin{soln}
	Possiamo definire l'insieme $\faktor A \sim$ per separazione nelle parti di $A$ come segue:
	\[ \faktor{A}{\sim} \Mydef \{B \in \ps(A) | \forall x \in A \; \forall y \in B \; x \sim y \leftrightarrow x \in B\}
		\]
	Osserviamo che per ogni $B, C \in \faktor A \sim$, vale che $B \cap C \ne \emptyset \iff B = C$, infatti, se esiste $x \in B \cap C$, allora $x \sim y$, $\forall y \in B$, e $x \sim z$, $\forall z \in C$.
	Da cui $w \in B \iff w \sim x \iff w \in C$ e quindi per l'arbitrarietà di $x$, vale $B = C$.\footnote{Essendo che ogni elemento, per quanto detto è in una classe di equivalenza di $\faktor{A}{\sim}$, si ha anche che $\bigcup \faktor{A}{\sim} = A$, dunque le classi di equivalenza sono disgiunte
	e la loro unione dà proprio l'insieme, pertanto si dirà che formano una \vocab{partizione} dell'insieme $A$.}\\
	Da quanto appena osservato segue quindi che ogni $x \in A$ appartiene ad una e una sola \vocab{classe di equivalenza} (gli elementi di $\faktor A\sim$), in quanto è sempre almeno in relazione con se stesso per riflessività, possiamo quindi definire $i_\sim$ come 
	la funzione da $A$ a $\faktor A \sim$ che manda $x$ nella sua classe di equivalenza. Naturalmente $i_\sim(x) = i_\sim(y)$ equivale al dire che le due classi di equivalenza sono la stessa, dunque per definizione
	si ottiene proprio che $x \sim y$. Inoltre $i_\sim$ è surgettiva in quanto in ogni classe di equivalenza di $\faktor A \sim$ c'è almeno un elemento (per la riflessività delle relazioni di equivalenza), la cui immagine via $i_\sim$ dà appunto la classe.
\end{soln}

\begin{exercise}[Primo teorema di ``omomorfismo'', per insiemi]
	Data una relazione di equivalenza $\sim$ su $A$ e $f : A \rightarrow B$, affinché esista la funzione $\widetilde{f}: \faktor{A}{\sim} \rightarrow B$ tale che $f = \widetilde{f} \circ i_\sim$,
	è necessario e sufficiente che $\forall x,y \in A \; x \sim y \rightarrow f(x) = f(y)$.
\end{exercise}

\begin{soln}
	Osserviamo che\footnote{Per essere formalissimi, staremmo usando che $f = \widetilde{f} \circ i_\sim \iff f(x) = (\widetilde{f} \circ i_\sim)(x)$, $\forall x \in A$, ovvero
	l'estensionalità per funzioni vista in un'osservazione precedente.} $f(x) = (\widetilde{f} \circ i_\sim)(x)$, $\forall x \in A$ se e solo se $f(x) = \widetilde{f}(i_\sim(x))$, ora ciò equivale al fatto che 
	l'immagine dell'elemento $x \in A$ al LHS è uguale a quella della classe di equivalenza (che è un sottoinsieme di $A$) $i_\sim(x)$ tramite $\widetilde{f}$ al RHS. Per rispettare la relazione richiesta (che sarebbe poi la commutatività di un diagramma)
	possiamo definire $\widetilde{f}(C)$, $C \in \faktor{A}{\sim}$, come $f(z)$ per un qualunque $z \in C$.\\ Ora ci basta osservare che questa è una buona definizione, e lo è in quanto tutti gli elementi in $C$ sono in relazione $\sim$ tra loro e per ipotesi tale relazione 
	è che la loro immagine via $f$ sia la stessa, pertanto $f(x) = f(y)$, $\forall x,y \in C$. Infine, poiché $\forall x \in A \; x \in i_\sim(x)$, si ha proprio che $\widetilde{f}(i_\sim(x)) = f(x)$. Abbiamo quindi dimostrato che l'uguaglianza iniziale è vera 
	se $\sim$ è definita come nelle ipotesi, osserviamo che se tale uguaglianza funziona, allora due elementi sono in relazione via $\sim$ se e solo se hanno la stessa immagine. Infatti, si avrebbe che:
	\[ \begin{split}
		f(x) = f(y) &\iff \widetilde{f}(i_\sim(x)) = \widetilde{f}(i_\sim(y)) \\
					&\iff i_\sim(x) = i_\sim(y) \\
					&\iff x \sim y
	\end{split}
		\]
	dove la prima equivalenza è l'assunto, la seconda è la definizione di $\widetilde{f}$ (che è una bigezione tra $\faktor A\sim$ e $\Imm(f)$, per questo abbiamo usato l'iniettività), mentre l'ultima equivalenza è la definizione di classi di equivalenza.
\end{soln}